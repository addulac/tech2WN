\section{Mixed-membership stochastic block models and (un)weighted graphs}
\label{sec:model}

As usual, we consider here that a network is represented by a graph $G=(V,E)$ where $V$ is the set of nodes such that $N=|V|$ and $E$ the set of edges. We furthermore consider the matrix $Y=(y_{ij})_{1 \le i,j \le N}$ such that $y_{ij}=0$ if $(i,j) \notin E$ and $y_{ij} \in \mathbb{Z}_{>0}$ otherwise. When the network is unweighted, $y_{ij} \in \{0,1\}$ and $Y$ is the adjacency matrix.
\subsection{Weighted version of MMSB (WMMSB)} 
Mixed-membership stochastic block models extend stochastic block models \cite{airoldi2009mixed} by allowing nodes to "belong" to several blocks (or classes) through a given (usually Dirichlet) probability distribution. Prior to generate a link between two nodes, a particular class is selected for each node. The link is then generated according to a probability distribution $F$, sometimes referred to as the \textit{kernel} distribution, that depends on the selected classes. The generative process behind such models can be summarized as: (a) For each node $i$, draw $\theta_i \sim \textrm{Dir}(\alpha)$, where $\theta_i$ and $\boldsymbol{\alpha}$ are $K$-dimensional vectors, $K$ denoting  the number of classes considered; (b) Generate two sets of latent class memberships, $Z_\rightarrow = \{z_{i\rightarrow j} \sim \textrm{Cat}(\theta_i),  1 \le i,j \le N\}$ and $Z_\leftarrow = \{z_{i\leftarrow j} \sim \textrm{Cat}(\theta_j),  1 \le i,j \le N\}$, with categorical draws; (c) Generate or not a link between two nodes $(i,j)$ according to $y_{ij} \sim F(\phi_{z_{i \rightarrow j}z_{i \leftarrow j}})$, where $F$ is a distribution in the exponential family and $\phi_{z_{i \rightarrow j}z_{i \leftarrow j}}$ a variable, usually drawn from a conjugate distribution, that represents the relations between classes. A standard choice for $F$ is the Bernoulli distribution, $\phi$ being the conjugate Beta distribution: $y_{ij} \sim \textrm{Bern}(\phi_{z_{i \rightarrow j}z_{i \leftarrow j}}), \, \phi_{kk'} \sim \textrm{Beta}(\lambda_0, \lambda_1)$. We refer to this model as the MMSB model.

A natural way to model weights that correspond to positive integers is to use Poisson distributions. Considering a conjugate Gamma distribution for $\phi$, we define a first weighted extension of MMSB, which we will refer to as WMMSB, through:
%
\begin{align*} \label{eq:generative}
\theta_i & \sim \textrm{Dir}(\alpha), \,\, z_{i\rightarrow j} \sim \textrm{Cat}(\theta_i), \,\, z_{i\leftarrow j} \sim \textrm{Cat}(\theta_j), \\
y_{ij} &\sim \textrm{Poi}(\phi_{z_{i \rightarrow j}z_{i \leftarrow j}}), \,\, \phi_{kk'} \sim \textrm{Gamma}(r, \frac{1-p}{p}),
\end{align*}
%
where $r$ and $\frac{1-p}{p}$ are the shape and rate parameters of the Gamma distribution. Note that both directed and undirected graphs can be considered with the above formulation, the matrix $\Phi = (\phi_{kk'})_{k,k' \in \{1,..,K\}^2}$ being symmetric for undirected graphs.

The Poisson-Gamma combination in WMMSB is interesting as it directly leads to: $y_{ij}|_{Z} \sim \textrm{NB}(r,p)$, where $\textrm{NB}$ denotes the negative binomial distribution. This latter distribution allows one to represent overdispersed count data and has been used in different contexts.
% Another advantage of the Poisson distribution is the fact that a truncated version of it, the Zero Truncated Poisson distribution, can easily be obtained by dividing the probability mass of the original distribution by $(1 - e^{\phi_{z_{i \rightarrow j}z_{i \leftarrow j}}})$. This latter version can directly be used to predict weights on existing links in networks, as in that case $y_{ij} > 0$. 
Lastly, by marginalizing over the variables $z_{i \rightarrow j}$ and $z_{i \leftarrow j}$, which take values in $\{1, \cdots, K\}$, one obtains:
 %
 \begin{equation}\label{eq:PoiCombin}
 y_{ij}|_{\Theta,\Phi} \sim \sum_{1 \le k,k' \le K} \theta_{ik} \theta_{jk'} \textrm{Poi}(\phi_{kk'}).
 \end{equation}
%

\subsection{A Beta-Gamma augmentation (WMMSB-bg)} 

The generative process for WMMSB defined above assumes that the parameters of the Poisson distributions used to generate links are drawn from the same Gamma distribution. Having a unique prior over these parameters however limits the ability of the model to capture the variance in the relations between the latent classes. Hierarchical extensions can be used here to have a better representation of the classes and the relations between them through class-dependent shape and rate parameters in the above Gamma distribution. We propose here to model the class-dependent shape parameter with another Gamma distribution and the class-dependent rate parameter with a Beta prior:
%
\begin{gather*}
r_{kk'} \sim \textrm{Gamma}(c_0r_0, 1/c_0), \, p_{kk'} \sim \textrm{Beta}(c\epsilon, c(1-\epsilon)).
%\phi_{kk'} \sim \textrm{Gamma}(r_{kk'}, \frac{p_{kk'}}{1-p_{kk'}})
\end{gather*}
%
The variable $y_{ij}$ is again distributed according to a negative binomial distribution, of the form:
%
\begin{equation}\label{eq:yNB}
y_{ij}|_{Z} \sim \textrm{NB}(r_{z_{i \rightarrow j} z_{i \leftarrow j}},p_{z_{i \rightarrow j} z_{i \leftarrow j}}).
\end{equation}
%
As one can note, and contrary to WMMSB, the parameters of the negative binomial distribution depend this time on the classes selected for each node, meaning that classes now play a prominent role in the model. We will refer to this model as WMMSB-bg.

The above extension exploits again the conjugacy of the distributions considered and is reminiscent of the Beta-Gamma-Gamma-Poisson model \cite{zhou2012beta} and the Gamma-Negative Binomial process \cite{zhou2015negative}. However, as for most hierarchical Bayesian models, exact inference is intractable and one must resort to approximate inference. The next section describes the variational inference scheme we have followed for that.

\section{Inference}
\label{sec:inference}

%One drawback of doing inference with Gibbs sampling for MMSB models, as is standard, is the quadratic complexity in the number of nodes. If one wants to use MMSB models and its weighted extensions on large networks, one needs to rely on other techniques, as variational inference.
Variational Inference (VI) and MCMC are two popular methods for learning the parameters of Bayesian models that (roughly speaking) realize a trade-off between time complexity (in particular wrt the size of the training set), one of the main advantages of VI, and accuracy in approximating the posterior distribution, one of the main advantages of MCMC. We have chosen the former as we are interested in a version of our model that scales to large datasets \footnote{Several studies have tried to scale MCMC to large datasets, using either, as far as we know, divide-and-conquer or subsampling approaches. But, as noted in \cite{Bardenet2017}, divide-and-conquer approaches have yet to solve a recombination method (on MCMC estimates computed on subsets of the data), while subsampling approaches only guarantee control of the approximation of the true posterior distribution in few situations that are rarely satisfied in practice, thus losing the main advantage of MCMC methods.}.
Collapsed variational Bayes inference presents the advantage, over standard variational inference, to rely on weaker assumptions and has proven to be efficient on the latent Dirichlet allocation model \cite{teh2007collapsed}. Recent advances in stochastic variational inference \cite{hoffman2013stochastic}, notably based on well-designed sampling techniques \cite{gopalan2013efficient,kim2013efficient}, have furthermore shown that it is possible to speed-up (collapsed) variational inference with online updates based on minibatches.
%so that the overall complexity is only linear in the number of edges and quadratic in the number of latent classes.
Coupling collapsed and stochastic variational inference thus leads here to an efficient inference method that can be used on large networks.
%Such approaches are thus very well adapted to online settings, in which links are observed during certain time intervals, over sparse networks (real world networks are most of the time very sparse \cite{barabasi_burst}).
%We combine here SVI with Collapsed Variational Bayes (CVB) inference \cite{teh2007collapsed} that relies on a weaker assumption on the variational distribution than standard variational inference.

We first provide below the results obtained through collapsed variational inference for all the above models. A detailed derivation of these results is given in Appendix~\ref{app:collap}. We then describe how stochastic variational inference is used on these models.

\subsection{Collapsed variational inference}

In the remainder, we use the notation $n^{-ij}$ to indicate that the superscript $ij$ is excluded from the underlying count variable, and $n_{\bm{.}}$ to indicate a sum over the dotted subscript index. Furthermore, $\Pi$ will denote the model parameters : $\Pi = (\Theta,\Phi,Z)$ for MMSB and WMMSB, where $\Theta$ is a $K \times N$ matrix, $\Phi$ a $K \times K$ matrix and $Z$ an $N \times N$ matrix, and $\Pi = (\Theta,\Phi,Z,R,P)$ for WMMSB-bg, where $R$ and $P$ are $K \times K$ matrices. Lastly, $\Omega$ will denote the hyperparameters ($\Omega = (\alpha,\lambda_0,\lambda_1)$ for MMSB, $\Omega = (\alpha,r,p)$ for WMMSB and $\Omega = (\alpha, c_0, r_0, c, \epsilon)$ for WMMSB-bg).

From Jensen's inequality, for any distribution $q$, one has: $\log p({Y} | \Omega) \ge \E_{q}[\log p({Y}, \Pi\ | \Omega)] + \textrm{H}[q(\Pi)]$,
%%
%\begin{equation*}
%\log p({Y} | \Omega) \ge \E_{q}[\log p({Y}, \Pi\ | \Omega)] + \textrm{H}[q(\Pi)]
%\end{equation*}
%%
where $\textrm{H}$ denotes the entropy. The goal of variational inference is then to find $q$ that maximizes the right-hand side of the above inequality, usually referred to as the Evidence Lower BOund (ELBO). In its collapsed version, following \cite{teh2007collapsed}, one weakens the mean-field assumption made over the variational distribution, leading to, for MMSB and WMMSB: $q(\Pi) = q(\theta, \Phi | Z) q(Z)$.
%%
%\begin{equation*}
%q(\Pi) = q(\Theta, \Phi | Z) q(Z)
%\end{equation*}
%%
For all $(i,j)$, $q(z_{i \rightarrow j}, z_{i \leftarrow j}|\gamma_{ij})$ follows a multinomial distribution with parameters $\gamma_{ijkk'}, \, (k,k') \in \{1, \cdots, K\}^2$. The evidence is then lower bounded by:
%
\begin{equation*}
\log p({Y}|\Omega) \geq \underbrace{\E_{q}[\log p({Y}, Z)] + \textrm{H}[q(Z)]}_{\L_Z}.
\end{equation*}
%
The derivation of the collapsed variational updates is obtained by maximizing the ELBO w.r.t $\gamma_{ijkk'}$:
%
\begin{align*}
\frac{\partial \L_Z}{\partial \gamma_{ijkk'}} &= \frac{\partial }{\partial \gamma_{ijkk'}} \left( \sum_{{Z}^{-ij}}\sum_{k_1=1}^K\sum_{k_2=1}^K  q({Z}^{-ij}) \gamma_{ijk_1 k_2} \right. \\
& (\log p({Y}, {Z}^{-ij}, z_{i\rightarrow j}=k_1, z_{i\leftarrow j}=k_2|\Omega)+ \\
& \left. \log q({Z}^{-ij}, z_{i\rightarrow j}=k_1, z_{i\leftarrow j}=k_2) ) \frac{}{} \right) \\
& = E_{q({Z}^{-ij})}[ p({Y}, {Z}^{-ij}, z_{i\rightarrow j}=k, z_{i\leftarrow j}=k'|\Omega))] \\
& + H[{Z}^{-ij}] -\log(\gamma_{ijkk'}) +1.
\end{align*}
%
By equating this derivative to zero, one obtains the following update:
\begin{align} \label{eq:maximization}
&\gamma_{ijkk'} \propto \nonumber\\
&\exp E_{q({Y}^{-ij})} [\log P(z_{i\rightarrow j}=k, z_{i\leftarrow j}=k' | {Y}^{-ij}, {Y}^{-ij}, \Omega)],
\end{align}
%
with  $P(z_{i\rightarrow j}=k, z_{i\leftarrow j}=k' | {Y}^{-ij}, {Z}^{-ij}, \Omega)$ being the collapsed Gibbs update of WMMSB. They take the form:
%
\begin{align*}
P(z_{i\rightarrow j} & = k, z_{i\leftarrow j}=k' |{Y}^{-ij}, {Z}^{-ij}, \Omega) \\
& \propto (n_{\rightarrow ik}^{\theta^{-j}} + \alpha_k) (n_{\leftarrow jk}^{\theta^{-i}} + \alpha_{k'}) \\
& \mathrm{NB}\left(y_{ij}; n^{{Y}^{-ij}}_{kk'} + r, \frac{p}{p\,n^{\Phi^{-ij}}_{\bm{.}kk'} + 1} \right),
\end{align*}
%
with count statistics given by:

%\begin{align} \label{eq:sss}
%    n^{\theta}_{\rightarrow ik} &= \sum_{j, k'} \gamma_{ijkk'}        & n^{\theta}_{\leftarrow jk'} &= \sum_{i, k} \gamma_{ijkk'}  \nonumber \\
%    n^{\Phi}_{xkk'} &= \sum_{ij:y_{ij}=x} \gamma_{ijkk'}  & n^{{Y}}_{kk'} &= \sum_{ij} y_{ij}\gamma_{ijkk'}
%\end{align}

\begin{align*}                                                                                                                                        
&n^{\theta}_{\rightarrow ik} = \sum_j \delta(\zij=k), \\
&n^{{Y}}_{kk'} = \sum_{ij} y_{ij}\delta(\zij=k, \zji=k'), \\
&n^{\Phi}_{\bm{.}kk'} = \sum_{ij} \delta( \zij=k, \zji=k').
\end{align*}   

By applying a first order Taylor expansion on Eq.~\eqref{eq:maximization}, following \cite{teh2007collapsed}, one obtains:
%
\begin{align*}
\gamma_{ijkk'} \propto & (E_{q({Z}^{-ij})}[n_{\rightarrow ik}^{\theta^{-j}}] + \alpha_k) (E_{q({Z}^{-ij})}[n_{\leftarrow jk}^{\theta^{-i}}] + \alpha_{k'}) \\
& \mathrm{NB}\left(y_{ij}; E_{q({Z}^{-ij})}[n^{{Y}^{-ij}}_{kk'}] + r,  p' \right),
\end{align*}
%
with:
%
\[
p' = \frac{p}{p\,E_{q({Z}^{-ij})}[n^{\Phi^{-ij}}_{\bm{.}kk'}] + 1}.
\]
%
Finally, using a Gaussian approximation (as in \textit{e.g.} \cite{asuncion2009smoothing}), one obtains, setting $E_{q({Z}^{-ij})}[n_{\rightarrow ik}^{\theta^{-j}}] = N^{\theta}_{\rightarrow ik}$, $E_{q({Z}^{-ij})}[n_{\leftarrow jk}^{\theta^{-i}}] = N^{\theta}_{\leftarrow jk'}$, $E_{q({Z}^{-ij})}[n^{\Phi^{-ij}}_{\bm{.}kk'}] = N^{\Phi}_{xkk'}$ and $_{q({Z}^{-ij})}[n^{{Y}^{-ij}}_{kk'}] = N^{{Y}}_{kk'}$:
%
\begin{align} \label{eq:sss}
   N^{\theta}_{\rightarrow ik} &= \sum_{j, k'} \gamma_{ijkk'},       & N^{\theta}_{\leftarrow jk'} &= \sum_{i, k} \gamma_{ijkk'},  \nonumber \\
   N^{\Phi}_{xkk'} &= \sum_{ij:y_{ij}=x} \gamma_{ijkk'},  & E &= \sum_{ij} y_{ij}\gamma_{ijkk'}.
\end{align}
%
In this inference scheme, the parameters $\gamma_{....}$ are the local parameters while the count statistics $N^{.}_{...}$ represent the sufficient statistics and global counts. Finally, the model parameters can be recovered from their estimates as follows:
%
\begin{align*}
\hat \theta_{ik} &= \frac{N^{\theta}_{\rightarrow ik} + N^{\theta}_{\leftarrow ik} + \alpha_k}{2N + \alpha_{\bm{.}}}, \\
\hat \phi_{kk'}&=\begin{cases}
     \frac{N^{\Phi}_{1 kk'} + \lambda_1}{N^{\Phi}_{\bm{.}kk'} + \lambda_{\bm{.}}} & \textrm{ \textit{for MMSB}}, \\
    \frac{p(N^{Y}_{kk'} + r)}{N^{\Phi}_{\bm{.}kk'} - p + 1}  & \textrm{ \textit{for WMMSB}}.  % \\
%    \frac{\E_{q}[p_{kk'}](N^{Y}_{kk'} + \E_{q}[r_{kk'}])}{N^{\Phi}_{\bm{.}kk'} - \E_{q}[p_{kk'}] + 1}  & \textrm{ for WMMSB-bg}
    \end{cases}
\end{align*}

\subsubsection{Beta-Gamma augmentation} 

For WMMSB-bg, the derivation is slightly more complex and is fully detailed in Appendix~\ref{app:collap}. We just provide here the main steps of this derivation.

We consider the following collapsed variational distribution for WMMSB-bg: $q(\Pi) = q(\theta, \Phi|Z, {R}, P)q(Z)q({R})q(P)$,
%%
%\begin{equation*}
%q(\Pi) = q(\theta, \Phi|Z, {R}, {Y})q(Z)q({R})q({Y})
%\end{equation*}
%%
with ${R}=(r_{kk'}), P=(p_{kk'}), \, 1 \le k,k' \le K$. As before, $q(z_{i \rightarrow j}, z_{i \leftarrow j}|\gamma_{ij})$ is multinomial with parameter $\gamma_{ij}$.

The same development as above applies for the parameters $\gamma_{ijkk'}$, given here also by Eq.~\ref{eq:maximization}. Furthermore, the predictive link probability and $\hat \phi_{kk'}$ now take the form:
%
\begin{align*}
p(y_{ij} | {Y}^{-ij}, Z^{- ij}, \zij=k, \zji=k',\Omega) \sim \\
\mathrm{NB}\left(y_{ij}; N^{{Y}^{-ij}}_{kk'} + \E_{q}[r_{kk'}], \frac{\E_{q}[p_{kk'}]}{\E_{q}[p_{kk'}]\,N^{\Phi^{-ij}}_{\bm{.}kk'} + 1} \right),
\end{align*}
%
and:
%
\begin{align*}
\hat \phi_{kk'} = \frac{\E_{q}[p_{kk'}](N^{Y}_{kk'} + \E_{q}[r_{kk'}])}{N^{\Phi}_{\bm{.}kk'} - \E_{q}[p_{kk'}] + 1}.
\end{align*}
%
Setting $q({P}) = p({P}|{Y},Z,\Omega)$ where $p$ is the true distribution and exploiting the conjugacy of the Beta and the negative binomial distributions leads to a Beta distribution for $p_{kk'}$: $p_{kk'} \sim \textrm{Beta}(c\epsilon + N^{Y}_{kk'}, c(1-\epsilon) + N^\Phi_{kk'}\E_{q}[r_{kk'}])$,
%%
%\begin{equation} \label{eq:pk_update}
%p_{kk'} \sim \textrm{Beta}(c\epsilon + N^{Y}_{kk'}, c(1-\epsilon) + N^\Phi_{kk'}\E_{q}[r_{kk'}])
%\end{equation}
%%
so that:
\[
\E_{q}[p_{kk'}] = \frac{c\epsilon + N^{Y}_{kk'}}{c\epsilon + N^{Y}_{kk'} + c(1-\epsilon) + N^\Phi_{kk'}\E_{q}[r_{kk'}]}.
\]

Lastly, as for its true distribution, the variational distribution for $r_{kk'}$ is taken in the Gamma family:  $q(r_{kk'}) \sim \textrm{Gamma}(a_{kk'},b_{kk'})$. Even though $a_{kk'}$ can not be estimated explicitly, one only needs to have access to the expectation of $r_{kk'}$, that takes the following form:
\[
\E_{q}[r_{kk'}] = \frac{r_0c_0+N^{Y}_{kk'}}{c_0  -N^\Phi_{kk'}\log(1-p_{kk'})}.
\]
%
%\begin{equation}
%\E_{q}[r_{kk'}] = \frac{r_0c_0+N^{Y}_{kk'}}{c_0  -N^\Phi_{kk'}\log(1-p_{kk'})}
%\end{equation}

\subsection{Stochastic variational inference with stratified sampling}

Stochastic variational inference can then be used to optimize the collapsed ELBO through unbiased, yet noisy, estimates of its natural gradient computed on sampled data points. Different sampling strategies \cite{gopalan2013efficient,kim2013efficient} have been proposed for that purpose. Following the study in \cite{gopalan2013efficient}, we rely here on stratified sampling that allows one to control the number of links and non-links used in the inference process. For each node $i, \, 1 \le i \le N$, one first constructs a set, denoted $s_1^i$, containing all the nodes to which $i$ is connected to as well as $M$ sets of equal size, denoted $s_0^{i,m}, \, 1 \le m \le M$, each containing a sample of the nodes to which $i$ is not connected to\footnote{The sampling is here uniform over the nodes not connected to $i$ with replacement; sampling without replacement led to poorer results in our experiments.}. We will denote by $S_0^i$ the set of all $s_0^{i,m}$ sets. The sets thus obtained, for all nodes, constitute minibatches that can be sampled and used to update the global counts in Eq.~\ref{eq:sss}. The combined scheme is summarized below:
%
\begin{enumerate}
\item Sample a node $i$ uniformly from all nodes in the graph; with probability $\frac{1}{2}$, either select $s_1^i$ or any set from $S_0^i$ (in the latter case, the selection is uniform over the sets in $S_0^i$). We will denote by $s_i$ the set selected and by $|s_i|$ its cardinality.
\item For each node $j \in s_i$, compute $\gamma_{ijkk'}$ through Eq.~\ref{eq:maximization} and intermediate global counts according to:
{\scriptsize
\begin{align*}
&\hat N^{\theta}_{\rightarrow ik} \mathrel{+}= \frac{1}{|s_i|} \frac{1}{Cg(s_i)} \sum_{k'} \gamma_{ijkk'}, \,\, \hat N^{\theta}_{\leftarrow jk'} = \frac{1}{Cg(s_i)} \sum_{k} \gamma_{ijkk'}, \\
&\hat N^{\Phi}_{xkk'} \mathrel{+}= \frac{1}{|s_i|} \frac{1}{Cg(s_i)} \gamma_{ijkk'}, \,\, \hat N^{{Y}}_{kk'} \mathrel{+}= \frac{1}{|s_i|} \frac{1}{Cg(s_i)} \gamma_{ijkk'} y_{ij}
\end{align*}
}
%\begin{align}
%\hat N^{\theta}_{\rightarrow ik} += \frac{1}{|s_i|} \frac{1}{Cg(s_i)} \sum_{k'} \gamma_{ijkk'}, & \,\, \hat N^{\theta}_{\leftarrow jk'} = \frac{1}{Cg(s_i)} \sum_{k} \gamma_{ijkk'} \nonumber \\
%\hat N^{\Phi}_{xkk'} += \frac{1}{|s_i|} \frac{1}{Cg(s_i)} \gamma_{ijkk'}, & \,\, \hat N^{{Y}}_{kk'} += \frac{1}{|s_i|} \frac{1}{Cg(s_i)} \gamma_{ijkk'} y_{ij} \nonumber
%\end{align}
where $C$ is a constant that is $2$ for undirected graphs and $1$ for directed graphs and $g(s_i) = \frac{1}{Nm}$ if $s_i \in S_0^i$ and $\frac{1}{N}$ otherwise.
\item Update of the global counts (online version of Eq.~\ref{eq:sss}): 
{\scriptsize
\begin{align*}
N^{\theta}_{\rightarrow ik} &\leftarrow (1 - \rho^{i,\theta}_t) N^{\theta}_{\rightarrow ik} + \rho^{i,\theta}_t \hat N^{\theta}_{\rightarrow ik}, \\
N^{\theta}_{\leftarrow jk'} &\leftarrow (1 - \rho^{i,\theta}_t) N^{\theta}_{\leftarrow jk'} + \rho^{i,\theta}_t \hat N^{\theta}_{\leftarrow jk'}, \\
N^{\Phi}_{xkk'} &\leftarrow (1 - \rho^{\Phi}_t) N^{\Phi}_{xkk'} + \rho^{\Phi}_t \hat N^{\Phi}_{xkk'}, \\
N^{{Y}}_{kk'} &\leftarrow (1 - \rho^{{Y}}_t) N^{{Y}}_{kk'} + \rho^{{Y}}_t \hat N^{{Y}}_{kk'}
\end{align*}
}
%\begin{align}
%N^{\theta}_{\rightarrow ik} \leftarrow (1 - \rho^{i,\theta}_t) N^{\theta}_{\rightarrow ik} + \rho^{i,\theta}_t \hat N^{\theta}_{\rightarrow ik}, & \,\,\, N^{\theta}_{\leftarrow jk'} \leftarrow (1 - \rho^{i,\theta}_t) N^{\theta}_{\leftarrow jk'} + \rho^{i,\theta}_t \hat N^{\theta}_{\leftarrow jk'} \nonumber \\
%N^{\Phi}_{xkk'} \leftarrow (1 - \rho^{\Phi}_t) N^{\Phi}_{xkk'} + \rho^{\Phi}_t \hat N^{\Phi}_{xkk'}, & \,\,\, N^{{Y}}_{kk'} \leftarrow (1 - \rho^{{Y}}_t) + \rho^{{Y}}_t \hat N^{{Y}}_{kk'} \nonumber
%\end{align}
\item $\rho^*_t = \frac{1}{(\tau +t)^\kappa}$ with $\kappa \in (0.5, 1]$.
\item Go back to step 1 till convergence.
\end{enumerate}
%
As one can note, the intermediate global counts correspond to a restriction, on minibatches, of the complete computation given in Eq.~\ref{eq:sss}. The value of $C$ is due to the fact that in undirected networks, each edge can be seen twice. The terms $\frac{1}{|s_i|}$ and $\frac{1}{Cg(s_i)}$ serve as a normalization in the gradient-like updates of the global counts (as there are more non-links than links, each non-link minibatch, representing a smaller fraction of the non-links, leads to more conservative updates). The "gradient steps" $\rho^*$ are discussed below (Robbins-Monro condition).

Lastly, to be able to efficiently compute such quantities as $N^{\Phi^{-ij}}$ used for the computation of the link probability, one needs to store in memory, for each pair of nodes $(i,j)$, a $K \times K$ matrix, which is not feasible for large networks. Thus, following \cite{foulds2013stochastic}, we replace here $N^{\Phi^{-ij}}$ by $N^{\Phi}$, which amounts to assume that the contribution of each individual pair of nodes is negligible compared to all other pairs, a reasonable assumption when the network is large. %The complete code for the inference will be made available in GitHub upon acceptance.

\subsubsection{Robbins-Monro condition} 

\begin{table*}[t]
\tiny
\def\arraystretch{1} % 1 is the default, change whatever you need
	
\caption{Datasets networks used to train the models. Type A is for co-authorship, type C is for communication, type H is for hyperlinks and type L is for lexical network.}

\begin{tabular}{lrrrrcrrrr}
%\Xhline{2\arrayrulewidth}
\toprule
 Datasets     &   Nodes &   Edges &   Density & Directed  &    Diameter &   \multicolumn{3}{c}{Weights}  	& type     \\
 \cmidrule(l){7-9}  &   &   	  &   		  & 		  &  		   	&  mean & std  & max             \\
%\hline
\midrule
 astro-ph      &   16706 &  121251 &     0.001    & False    &       14 &   1.8& 3.3 & 306      & A  \\
 cond-mat      &   16726 &   47594 &     0.000    & False    &       18 &   3.1& 7.2& 544      & A  \\
 hep-th        &    8361 &   15751 &     0.000    & False    &        1 &   5.2& 16& 1226      & A  \\
 netscience    &    1589 &    2742 &     0.002    & False    &        2 &   2.2& 1.9 &33       & A  \\
 manufacturing &     167 &    5783 &     0.209    & True     &        3 &  14.3& 44.9&1458    & C  \\
 fb\_uc        &    1899 &   20296 &     0.006    & True     &        4 &   2.8& 4.7& 98       & C  \\
 enron         &    87273  & 320154   &  0.0000   & True     &       15 &   3.4& 12.4&3904    & C  \\
 wiki-link     &  100312   & 887426   & 0.0001    & True     &       14 &   1.7& 3.0& 185      & H  \\
%\Xhline{2\arrayrulewidth}
\bottomrule
\end{tabular}


\label{table:corpus}
\end{table*}

The convergence of stochastic variational inference is guaranteed under the Robbin-Monro condition \cite{robbins1951stochastic} that imposes constraints on the gradient step, $\sum \rho_t = \infty$ and $\sum \rho_t^2 < \infty$ which can be obtained with $\rho_t = \frac{1}{(\tau +t)^\kappa}$ with $\kappa \in (0.5, 1]$. Thus, we maintain a gradient step for each of the global counts $\rho^\Phi$ and $\rho^{Y}$ accounting respectively for  $N^\Phi$ and $N^{Y}$. For $N^\theta$, we maintain individual gradient steps $\rho_i^{\theta}$ for $1\leq i\leq N$, following \cite{miller2009nonparametric}; this improved both convergence and prediction performance. Furthermore, to increase the speed of the inference, we update the global count $N^\Phi$ and $N^{Y}$ only after a minibatch round. For the global count  $N^\theta$, we update it after a burn-in period $T_{burnin}$ such that $T_{burnin} \leq |S|$.
%, which consists to update the class assignment of nodes after observing a bunch of dyad.
This heuristic provides a trade-off between updating the global statistics after each observation, which slows down the inference and may result in bad local optima, and updating them only after minibatches that are potentially large (proportional to the number of nodes).

%%% Two more point exists:
% * the time step is increased with the minibatch size,
% * the intermiatade graidnet is normalized by the minibatch size.

%Our SCVI algorithm is summarized in pseudo-code \ref{algo:scvb}.
%
%\begin{algorithm}
%\KwIn{Random initialization of $N^\theta, N^\Phi, N^{Y}$.}
%\KwOut{$\thetah, \Phih$.}
%\Begin{
%$t \gets 0$ \\
%\While{Convergence criteria not met}{
%    Sample a minibatch $S_t$ from $h(S)$. \\
%    \ForEach{$i,j \in S_t$}{
%        Maximize local parameters $\gamma_{ij}$ from \eqref{eq:maximization}.\\
%        \If{burn-in finished}{
%            Compute intermediate gradient $\hat N^\theta$ from \eqref{eq:sss}.\\
%            Update global statistic $N^{\theta (t)}$.\\
%            Update gradient step $\rho^\theta_t$.\\
%        }
%    }
%
%    Compute intermediate gradient $\hat N^\Phi$ and $\hat N^Y$  from \eqref{eq:sss}.\\
%    Update global statistics $N^{\Phi (t)}$ and $N^{Y (t)}$.\\
%    Update gradient steps $\rho^\Phi_t$ and $\rho^\theta_t$.\\
%	Sample $P$ and $R$ from \eqref{eq:pk_update} \eqref{eq:rk_update}.\\
%    $t \gets t + 1$ .
%    }
%}
%\caption{SCVI pseudo-code.}
%\label{algo:scvb}
%\end{algorithm}

