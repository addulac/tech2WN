%\documentclass[runningheads]{llncs}

\documentclass[letterpaper]{article}
\usepackage{uai2020}
\usepackage[margin=1in]{geometry}
\usepackage{ulem}

\usepackage{times}


%\usepackage[cmex10]{amsmath, mathtools}
\usepackage{amsmath}
%\usepackage[fleqn]{amsmath}
%\usepackage{amssymb,amsbsy,amsfonts,amsthm}
\usepackage{amssymb,amsbsy,amsfonts}
\usepackage{bm}
\usepackage{enumerate}
\usepackage{url}
\usepackage[ruled,vlined]{algorithm2e}
\usepackage{fancyvrb}
\usepackage{yfonts}
\usepackage{multirow}
\usepackage{multicol}
\usepackage{adjustbox}
%\usepackage[margin=6.5em]{geometry}
\usepackage{makecell} % thicker table separator
\usepackage{booktabs}
\usepackage{listings}
\lstset{basicstyle=\ttfamily\color{blue}\scriptsize}
\usepackage{color}

\usepackage{subfigure}
\usepackage{wrapfig}
\usepackage{tikz}
%\input{../tikz.conf}

\usetikzlibrary{bayesnet}

%%%%%%%%%%% Box 
\usepackage{calc}%    For the \widthof macro
\usepackage{xparse}%  For \NewDocumentCommand
\newcommand{\tikzmark}[1]{\tikz[overlay,remember picture] \node (#1) {};}


%\input{./header.tex}
%%%%%%%%%% Math
\renewcommand{\text}{\textnormal}
%\newcommand{\pr}{\mathbf{p}}
\newcommand{\pr}{p}
\newcommand{\p}{p}
\newcommand{\E}{\mathbb{E}}
\newcommand{\divkk}{\mathbb{K}}
\newcommand{\entropy}{\mathbb{H}}
\newcommand{\gem}{\mathrm{GEM}}
\newcommand{\Mult}{\mathrm{Mult}}
\newcommand{\DP}{\mathrm{DP}}
\newcommand{\IBP}{\mathrm{IBP}}
\newcommand{\M}{\mathcal{M}}
\newcommand{\V}{\mathcal{V}}
\newcommand{\N}{\mathcal{N}}
\newcommand{\D}{\mathcal{D}}
\renewcommand{\L}{\mathcal{L}}
\newcommand{\mat}[1]{\mathbf{#1}}
\newcommand{\unit}{1\!\!1}
\newcommand{\zij}{z_{i\rightarrow j}}
\newcommand{\zji}{z_{i\leftarrow j}}
\newcommand{\Thetah}{\hat\Theta}
\newcommand{\Phih}{\hat\Phi}
\newcommand{\thetah}{\hat\theta}
\newcommand{\phih}{\hat\phi}

\newcommand{\Bs}[1]{\boldsymbol{#1}} % vector
\newcommand{\B}[1]{\mathbf{#1}} % vector

\newcommand\mms[1]{\vcenter{\hbox{$\scriptstyle #1$}}}

%\renewcommand{\Phi}{\mat{\Phi}}


%\date{avril 2015}

%\newtheorem{definition}{Definition}[section]
%\newtheorem{proposition}{Proposition}[section]
%\newtheorem{theorem}{Theorem}[section]
%\newtheorem{corollary}{Corollary}[section]
%\newtheorem{proof}{Proof}[section]


\begin{document}

\title{Mixed-Membership Stochastic Block Models for Weighted Networks}
	
\maketitle

\begin{abstract}
We address in this study the problem of modeling weighted networks through generalized stochastic block models. Stochastic block models, and their extensions through mixed-membership versions, are indeed popular methods for network analysis as they can account for the underlying classes/communities structuring real-world networks and can be used for different applications.
% This said, few such models have been developed for weighted networks. 
Our goal is to develop such models to solve the weight prediction problem that consists in predicting weights on links in weighted networks. To do so, we introduce new mixed-membership stochastic block models that can efficiently be learned through a coupling of collapsed and stochastic variational inference. These models, that represent the first \textit{weighted} mixed-membership stochastic block models to our knowledge, can be deployed on large networks comprising millions of edges. The experiments, conducted on diverse real-world networks, illustrate the good behavior of these new models.
%We propose an online learning algorithm designed to model binary, weighted, directed or undirected networks. It relies on a probabilistic framework inherited from the mixed-membership stochastic blockmodel. The inference combines the advantages of Variational Inference, in particular to derive stochastic gradient descent of the variational objectives which enables mini-batches updates, and Collapse Gibbs Sampling that weaken the assumption made by the classical mean-field approximation of the posterior distribution. We study the convergence of the inference and we evaluate the performance of the models on several real world networks. Our experiments show that our algorithm exhibits fast convergence and have competitive  results on links prediction task especially when the network is partially observed. Futhermore, we show that the weighted MMSB (WMMSB) with an Beta-Gamma priors proposed (WMMSB-bg) signifanctly improves link prediction on most of the weighted networks tested compared to MMSB.
\end{abstract}

\section{Introduction}

Relational data is widespread in many modern applications. From social networks to protein interactions, from physics to linguistics, all interacting objects can be represented as a graph where the objects are the nodes and the interactions the edges. The interest for modelling such networks has naturally increased with the availability of large datasets. Especially in the machine learning literature, that focused on link prediction, dimensionality reduction and data exploration tasks. One of the main challenge in this area is to be able to handle massive networks that emerge from the web. In this paper, we focus on networks that underpin some kind of social relationship such as collaboration or communication networks. In this context, we propose an online learning algorithm that we derived for both binary and count edge covariate, within the framework of Mixed-Membership Stochastic Blockmodel (MMSB).

%%% The type of networks that exists
%Complex networks are graphs that are used to represents real world relationnal information. In computer science, a major network is the web that connects a large amount of data. There is a large diversity in the type of data that can be interconnected, which ca be a set of people in a social plateform, a set of documents linked with hyperlinks, communication networks of email  or more recently a graph of transaction encoded in a blockchain. Outside the web an other important networks is the one made of the scientific collaborations.

%%% The Scalability problem => Sparse Network E/N**2 << 1
%The complexity (time and memory) of batch algorithm are polynomial for graph. Thus, the need of online algorithm, able to update a model as data become available is fundamental for scaling strategy. This can because of the temporality of the data or more simply because the data don't fit in memory. Another source of diversity in networks is the support for labelled and dynamic networks. In this paper we study and propose an algorithm based on latent models with rich priors who scale for complex and massive networks, with labeled edges (weighted networks), and that can be adapted to model the exchangeability of sequences of binary networks (temporal networks).

\section{Related work}
\label{sec:rl}

%\begin{itemize}
%\item on SBM and WSBM (Clauset/peixoto)
%\item on MMSB familly (Airoldi/Blei/Mimmo/Gopalan) and SVB
%\item on PFA (Poisson Factor Analysis) and Gamma Processes (Zhou etc).
%\item on SCVB (Foulds). (they show that scvb is similar to EM+map on made the links with online EM of (Cappé and Moulines)
%\end{itemize}
In social networks, the presence of a tie between two entities generally indicates that there is a relationship between them and in recent years, researchers proposed various methods to predict the presence or absence of such interaction. However, in many applications, it is the intensity of the relationship that is important. So, for example in epidemiology, it is not enough to know that two people have been in contact but it is also necessary to know the frequency of these contacts to know whether there is a risk of contagion or not. Similarly, in the field of transport, it is not enough to know that there is a motorway or an airline between two cities to analyze the population flows between them, it is also necessary to know the number of vehicles or passengers that goes from the first one to the other. In the field of economy and finance, to know if a company risks being absorbed by another it is not enough to know that the second took shares in the capital of the second but it is also necessary to know how much. This intensity of the relationship is generally modeled as a weight and the network is represented by a weighted graph. Unfortunately, in practice it is often easier to observe the interaction than to measure it. For this reason, there has been a lot of work devoted to link prediction \cite{Liben2007, Zaki2011,  Martinez2016}  %Wang2014
and less to edge weight prediction in social networks.


Like for link prediction, to solve the weight prediction, there are two main approaches, similarity-based and likelihood-based methods \cite{Lu2011}. The methods belonging to the first familly assume that the similarity between two nodes is correlated with their interaction strength. For instance, using this hypothesis, Zhao \textit{et al.} propose to predict missing links and their weights by extending unweighted similarity indices to weighted ones \cite{Zhao2015}. Zhu \textit{et al.} also introduce a method where the estimation of the weights is based on neighbor sets \cite{Zhu2016}.
The second approach formulates the problem in a probabilistic framework and includes extended versions of generative models, initially introduced to learn the network formation process, in particular, the stochastic block model (SBM) \cite{Karrer2011} \textit{CL :ref  Karrer ou Nowicki and Snijders?}.  
However, these models suffer from several limitations notably they consider that a node can  belong to only one class, which is not realistic in real networks and, secondly the inference algorithm is not able to tackle large size networks. The models and inference process presented in this paper aim to overcome these limits. 


The original MMSB model was proposed in \cite{airoldi2009mixed} with a variational inference scheme. The inference process was later extended with stochastic variational inference in \cite{gopalan2013efficient} and structured variational inference in \cite{kim2013efficient} for scalability purposes. Stochastic variational inference has been applied with a collapsed variational objective for the latent Dirichlet allocation model \cite{foulds2013stochastic}\sout{. To} \textit{and to} our knowledge, it is the first time that stochastic and collapsed variational inference are coupled in the context of stochastic block models. \textit{However, the previous models have been designed for link prediction and not for weight prediction.  }

\textit{Weighted versions of the stochastic block model have been intoduced firstly in \cite{mariadassou2010} and then in} \cite{aicher2014learning} who proposed WSBM. WSBM can be seen as a special case of our WMMSB model  in which nodes are constrained to belong to only one latent class \textit{whereas our model allows each node to belong to several classes}. More recently, an extended version of WSBM  model has been presented  in which different kernels can be used to model different types of weights \cite{peixoto2018nonparametric}. An efficient MCMC method is used for inference. If this type of models is interesting, it nevertheless relies again on the assumption that a node belongs to only one class, which may be inappropriate for real world networks. Furthermore, unlike MMSB models, the lack of a hierarchical prior structure does not allow one to rely on efficient non-parametric extensions (hence the use of costly model selection techniques for non-parametric versions). 

Similar to our model, count processes with Poisson distributions and Gamma conjugate priors have been studied \sout{by different authors} \textit{notably by Zhou et al.} \cite{zhou2012augment, zhou2015negative}. The relation of such processes with Negative Binomial processes is well-known and has been highlighted by these authors who applied  these processes for topic modeling, \sout{as} \textit{with} the Beta-Gamma-Gamma-Poisson model (EPM) (\cite{zhou2012beta}) that relies on MCMC inference. They also applied them for  overlapping community detection and link prediction \cite{zhou2015infinite}. The main difference between this model and WMMSB is that the former factorizes counts as Poisson variables of a sum of latent factors while, in WMMSB, counts are factorized as a convex sum of Poisson variable depending on class memberships.

Thus, the main theoretical contribution of this article is two-fold: firstly, we propose a mixed-membership stochastic block model, called WMMSB-bg, for weighted networks allowing nodes to belong to several classes, and secondly we \sout{show how to learn this model on large networks with a stochastic collapsed variational inference algorithm} \textit{design an efficient stochastic collapsed variational inference algorithm able to handle large size networks}.

%
%This work intersects with several groups of related works:
%
%First, the recent advance on Stochatistic Variationnal inference have made it possible to scale bayesian model to bigger dataset and to do online learning which enable a low memory footprint. This inference have first been proposed for topic modeling [1][2] before being adapted for the MMSB model with an adaptation to discover overllaping communities [3] [4],
%
%Nevertheless, the previous works only study the case of (undirected) binary networks.
%
%In [5] the author proposed an efficient inference algorithm for weigthed networks, based on a MCMC algorithms. The model is an extension of the SBM. Those models assumed that the class don't overllap. (I still have to dive into to understand how his inferecne works...) (does it allow online learning ? )
%
%Finally, SVB has been combined with CVB inference for topic modelling to propose a improoved over SVB. [6]
%
%This paper combines the different advantage of those works to propose a Online learning algorithm to models networks that can be weighted, with overllaping classes, directed or undirected.
%


%\section{Weighted Networks}
%
%Most of real networks exhibit a topology more complex than just binary relationship between nodes. Instead, the relations can be weighted and dynamic. For example, co-authorship networks can be constructed such that the edges covariate corresponds to the number of collaborations between the corresponding authors \cite{newman2001scientific}. In a communication network, the weight can be the number of messages sent from the sender to the receiver. In the web, documents are connected with hyperlinks where the count of those is for example used to construct the PageRank algorithm. Finally, in a linguistic network, a network of words can be built where the weight between two words is the number of times where they follow each other. Another useful case where weighted networks can be useful is temporal networks. For instance, in communication networks, messages are sent at a specific time, thus taking into account the number of messages send during a period allows to represent the strength of the relation over the time.
%
%
%A natural prior for count edge covariate is the Poisson distribution. In addition, it has several nice properties,
%%\cite{orbanz2015bayesian}
%
%\begin{itemize}
%\item{Additivity}: If $K_1 \sim \mathrm{Poi}(\alpha_1)$ and $K_2 \sim \mathrm{Poi}(\alpha_2)$ then,
%    \begin{equation*}
%        K_1 + K_2 = \mathrm{Poi}(\alpha_1 + \alpha_2)
%    \end{equation*}
%\item {Thinning}: The number of successes in a Poisson number of coin flips is Poisson, namely if $K \sim \mathrm{Poi}(\alpha)$ and $X_1,...,X_K \sim \mathrm{Bern}(p)$ then,
%    \begin{equation*}
%        \sum_{i=1}^K X_i = \mathrm{Poi}(p\alpha)
%    \end{equation*}
%\end{itemize}
%
%These two properties justify to build weighted networks datasets from sequence of either weighted graphs or binary graphs to feed a Poisson based model. In the rest of the paper we will assume that a network is represented by a graph $G=(V,E)$ where $V$ is the set of nodes such that $N=|V|$ and E the set of edges. We consider the adjacency matrix $Y=(y_{ij})_{ij\in N^2}$ such that $y_{ij}=0$ if $(i,j) \notin E$ and $y_{ij} > 0$ otherwise.
%
%In the rest of the paper, we will use the notation $n^{-ij}$ to indicate that the superscript $ij$ is excluded from the underlying count variable, and $n_{\bm{.}}$ to indicate a sum over the dotted subscript index.

\section{Mixed-membership stochastic block models and (un)weighted graphs}

As usual, we consider here that a network is represented by a graph $G=(V,E)$ where $V$ is the set of nodes such that $N=|V|$ and E the set of edges. We consider the adjacency matrix $Y=(y_{ij})_{ij\in N^2}$ such that $y_{ij}=0$ if $(i,j) \notin E$ and $y_{ij} > 0$ otherwise.

Mixed-Membership Stochastic Block (MMSB) models extend stochastic block models \cite{airoldi2009mixed} by allowing nodes to "belong" to several blocks (or classes) through a given (usually Dirichlet) probability distribution. Prior to generate a link between two nodes, a particular class is selected for each node. The link is then generated according to a probability distribution $F$ that depends on the selected classes. The generative process behind such models can be summarized as: (a) For each node $i$, draw $\theta_i \sim \textrm{Dir}(\alpha)$, where $\theta_i$ and $\alpha$ are $K$-dimensional vectors, where $K$ denotes  the number of classes considered; (b) Generate two sets of latent class memberships, $Z_\rightarrow = \{z_{i\rightarrow j} \sim \textrm{Cat}(\theta_i),  1 \le i,j \le N\}$ and $Z_\leftarrow = \{z_{i\leftarrow j} \sim \textrm{Cat}(\theta_j),  1 \le i,j \le N\}$, with categorical draws; (c) Generate or not a link between two nodes $(i,j)$ according to $y_{ij} \sim F(\phi_{z_{i \rightarrow j}z_{i \leftarrow j}})$, where $F$ is a distribution in the exponential family and $\phi_{z_{i \rightarrow j}z_{i \leftarrow j}}$ an associated (usually conjugate) distribution that represents the relations between classes. For unweighted graphs, $F$ is the Bernoulli distribution and $\phi$ its conjugate Beta distribution.

Many real networks nevertheless rely on graphs in which edges are naturally weighted. In co-authorship networks, for example, it is standard to consider edges weighted according to the number of collaborations between authors \cite{newman2001scientific}. In communication networks, the weight are based on the number of messages sent from the sender to the receiver. In text mining and natural language processing applications, it is also common to use word graphs in which edges are weighted on the basis of the number of times the words co-occur (in a sentence, paragraph or document). In all these cases, weights are integers that can naturally be modeled with Poisson distributions. Relying on its conjugate Gamma distribution for $\phi$, one finally obtains the following models, denoted MMSB for unweighted graphs and WMMSB for weighted graphs:
%
\[
\theta_i \sim \textrm{Dir}(\alpha), \,\, z_{i\rightarrow j} \sim \textrm{Cat}(\theta_i), \,\, z_{i\leftarrow j} \sim \textrm{Cat}(\theta_j)
\]
%
and:
%
\begin{align*} \label{eq:generative}
y_{ij} &\sim \textrm{Bern}(\phi_{z_{i \rightarrow j}z_{i \leftarrow j}}), &\phi_{kk'} &\sim \textrm{Beta}(\lambda_0, \lambda_1), & \textrm{  \textit{for unweighted graphs (MMSB)}} \\
y_{ij} &\sim \textrm{Poi}(\phi_{z_{i \rightarrow j}z_{i \leftarrow j}}), &\phi_{kk'} &\sim \textrm{Gamma}(r, \frac{p}{1-p}),    & \textrm{  \textit{for weighted graphs (WMMSB)}} 
\end{align*}
%
The choice made here for the Gamma distribution in WMMSB allows one to represent overdispersed count data as one has $y_{ij} \sim \textrm{NB}(r,p)$ \cite{zhou2012beta}, where $\textrm{NB}$ denotes the negative binomial distribution. Furthermore, the above models are valid for both directed and undirected graphs, the matrix $\Phi = (\phi_{kk'})_{k,k' \in \{1,..,K\}^2}$ being symmetric in the latter case.

%In the following we will denote the set of the model parameters $\Pi = \{ \Theta, \Phi, Z \}$ with $\Theta = (\theta_{ik})_{i,k \in \{1,..,N\}\times \{1,..,K\}}$ and the set of model hyper-parameters $\Omega$.

\subsection{Beta-Gamma augmentation}

The generative process for WMMSB defined above assumes that the parameters of the Poisson distributions used to generate links are drawn from the same Gamma distribution. Having a unique prior over these parameters however limits the ability of the model to capture the variance in the relations between the latent classes. Hierarchical extensions can be used here to have a better representation of the classes and the relations between them. Following the Beta-Gamma-Gamma-Poisson model \cite{zhou2012beta} and the Gamma-Negative Binomial process \cite{zhou2015negative}, we model here the rate parameter of the Gamma distribution used in WMMSB with a Beta prior and its shape parameter with another Gamma distribution of the form:

\begin{gather*}
r_{kk'} \sim \textrm{Gamma}(c_0r_0, 1/c_0) \qquad p_{kk'} \sim \textrm{Beta}(c\epsilon, c(1-\epsilon)) \\
\phi_{kk'} \sim \textrm{Gamma}(r_{kk'}, \frac{p_{kk'}}{1-p_{kk'}})
\end{gather*}

The hyperparameters of this model, denoted as bg-WMMSB, are: $\Omega = (\alpha, c_0, r_0, c, \epsilon)$. The variable $y_{ij}$ is again distributed according to a negative binomial distribution, of the form: $y_{ij}|_{Z} \sim \textrm{NB}(r_{z_{i \rightarrow j} z_{i \leftarrow j}},p_{z_{i \rightarrow j} z_{i \leftarrow j}})$. As one can note, and contrary to WMMSB, the parameters of the negative binomial distribution depend this time on the classes selected for each node, meaning that classes know play a prominent role in the model.

As for most hierarchical Bayesian model, exact inference is intractable and one must resort to approximate inference. In the next section we propose a stochastic collapsed variational inference algorithm for the above models (MMSB, WMMSB, bg-WMMSB) that can be used in online settings.

\section{Inference}

One drawback of the original inference method proposed for MMSB models is its quadratic complexity in the number of nodes. It is impossible to use such a method on large networks. Recent advances in Stochastic Variational Inference (SVI) \cite{hoffman2013stochastic}, notably based on well designed sampling techniques \cite{gopalan2013efficient,kim2013efficient}, address this challenge by enabling the control of the memory complexity with online/minibatch updates while keeping the time complexity linear in the number of edges present in the network. Such approaches are thus very well adapted to online settings, in which links are observed during certain time intervals, over sparse networks (real world networks are most of the time very sparse \cite{barabasi_burst}).

We propose to combine here SVI with Collapsed Variational Bayesian (CVB) inference \cite{teh2006collapsed} that relies on a weaker assumption on the variational distribution.

\subsection{Collapsed variational inference}

In the remainder, we use the notation $n^{-ij}$ to indicate that the superscript $ij$ is excluded from the underlying count variable, and $n_{\bm{.}}$ to indicate a sum over the dotted subscript index. Furthermore, $\Pi$ will denote the model parameters ($\Pi = (\Theta,\Phi,Z)$ for MMSB and WMMSB and $(\Theta,\Phi,Z,R,P)$ for bg-WMMSB) and $\Omega$ the hyperparameters ($\Omega = (\alpha,\lambda_0,\lambda_1)$ for MMSB, $(\alpha,r,p)$ for WMMSB and $(\alpha, c_0, r_0, c, \epsilon)$ for bg-WMMSB). From Jensen's inequality, for any distribution $q$, one has:
%
\begin{equation*}
\log p(Y | \Omega) \ge \E_{q}[\log p(Y, \Pi\ | \Omega)] + \textrm{H}[q(\Pi)]
\end{equation*}
%
where $\textrm{H}$ denotes the entropy. The goal of variational inference is then to find $q$ that maximizes the right-hand side of the above inequality, usually referred to as the Evidence Lower BOund (ELBO). In its collapsed version, following \cite{teh2006collapsed}, one weakens the mean-field assumption made over the variational distribution, leading to, for MMSB and WMMSB:
%
\begin{equation*}
q(\Pi) = q(\Theta, \Phi | Z) q(Z)
\end{equation*}
%
with $q(z_{i \rightarrow j}, z_{i \leftarrow j}|\gamma_{ij})$ being multinomial with parameter $\gamma_{ij}$. The evidence is then lower bounded by:
%
\begin{equation*}
\log p(Y|\Omega) \geq \underbrace{\E_{q}[\log p(Y, Z)] + \textrm{H}[q(Z)]}_{\L_Z}
\end{equation*}

Maximizing $\L_Z$ w.r.t $\gamma_{ijkk'}$ under a zero order Taylor expansion and a Gaussian approximation, following \cite{teh2006collapsed,asuncion2009smoothing}, yields the following updates, detailed in the supplementary material:
%
\begin{equation} \label{eq:maximization}
\gamma_{ijkk'} \propto (N_{\rightarrow ik}^{\Theta^{-j}} + \alpha_k) (N_{\leftarrow jk}^{\Theta^{-i}} + \alpha_{k'}) p(y_{ij} | Y^{\neg ij}, Z^{\neg ij}, \zij=k, \zji=k')
\end{equation}
%
where the elements $N^{\Theta}$ are defined in Eqs~\eqref{eq:sss}. Depending on the model considered, the predictive link distribution $p(y_{ij}|z_{i \rightarrow j}=k,z_{i \leftarrow j}=k')$ takes the following form:
%
\begin{align*}
p(y_{ij} | z_{i \rightarrow j} = k,z_{i \leftarrow j} = k') \begin{cases}
    = \left( \frac{ N^{\Phi^{-ij}}_{1 kk'} + \lambda_1}{N^{\Phi^{-ij}}_{\bm{.}kk'} + \lambda_{\bm{.}}}\right)^{y_{ij}} \left( 1- \frac{ N^{\Phi^{-ij}}_{1 kk'} + \lambda_1}{N^{\Phi^{-ij}}_{\bm{.}kk'} + \lambda_{\bm{.}}}\right)^{1-y_{ij}}  & \textrm{ for MMSB} \\
    \sim \mathrm{NB}\left(y_{ij}; N^{Y^{-ij}}_{kk'} + r, \frac{p}{p\,N^{\Phi^{-ij}}_{\bm{.}kk'} + 1} \right) & \textrm{ for WMMSB} % \\
%    \sim \mathrm{NB}\left(y_{ij}; N^{Y^{-ij}}_{kk'} + \E_{q}[r_{kk'}], \frac{\E_{q}[p_{kk'}]}{\E_{q}[p_{kk'}]\,N^{\Phi^{-ij}}_{\bm{.}kk'} + 1} \right) & \textrm{ for bg-WMMSB}
\end{cases}
\end{align*}

The different count statistics $N^*$ are estimated from the variational parameters $\gamma_{ijkk'}$ by:
%
\begin{align} \label{eq:sss}
    N^{\Theta}_{\rightarrow ik} &= \sum_{j, k'} \gamma_{ijkk'}        & N^{\Theta}_{\leftarrow jk'} &= \sum_{i, k} \gamma_{ijkk'}  \nonumber \\
    N^{\Phi}_{xkk'} &= \sum_{ij:y_{ij}=x} \gamma_{ijkk'}  & N^{Y}_{kk'} &= \sum_{ij} y_{ij}\gamma_{ijkk'}
\end{align}
%
In this inference scheme, $\gamma_{ij}$ are the \emph{local} parameters while the count statistics $N^*$ represent the \emph{global} parameters.  

Finally, the model parameters can be recovered from their estimates as follows:
%
\begin{align*}
\hat \theta_{ik} = \frac{N^{\Theta}_{\rightarrow ik} + N^{\Theta}_{\leftarrow ik} + \alpha_k}{2N + \alpha_{\bm{.}}} \qquad 
\hat \phi_{kk'}=\begin{cases}
     \frac{N^{\Phi}_{1 kk'} + \lambda_1}{N^{\Phi}_{\bm{.}kk'} + \lambda_{\bm{.}}} & \textrm{ for MMSB} \\
    \frac{p(N^Y_{kk'} + r)}{N^{\Phi}_{\bm{.}kk'} - p + 1}  & \textrm{ for WMMSB}  % \\
%    \frac{\E_{q}[p_{kk'}](N^Y_{kk'} + \E_{q}[r_{kk'}])}{N^{\Phi}_{\bm{.}kk'} - \E_{q}[p_{kk'}] + 1}  & \textrm{ for bg-WMMSB} 
    \end{cases}
\end{align*}

\subsubsection{Beta-Gamma augmentation}

For bg-WMMSB model, we consider the following collapsed variational distribution:
%
\begin{equation*}
q(\Pi) = q(\Theta, \Phi|Z, R, P)q(Z)q(R)q(P)
\end{equation*}
%
with $R=(r_{kk'})$ and $P=(p_{kk'})$, $k,k' \in \{1,..,K\}^2$. As before, $q(z_{i \rightarrow j}, z_{i \leftarrow j}|\gamma_{ij})$ is multinomial with parameter $\gamma_{ij}$. 

Following the same development as before, the parameters $\gamma_{ijkk'}$ are given by Eq.~\ref{eq:maximization}. Furthermore, $p(y_{ij}|z_{i \rightarrow j}=k,z_{i \leftarrow j}=k')$ and $\hat \phi_{kk'}$ now take the form:
%
\[
p(y_{ij} | z_{i \rightarrow j} = k,z_{i \leftarrow j} = k')  \sim \mathrm{NB}\left(y_{ij}; N^{Y^{-ij}}_{kk'} + \E_{q}[r_{kk'}], \frac{\E_{q}[p_{kk'}]}{\E_{q}[p_{kk'}]\,N^{\Phi^{-ij}}_{\bm{.}kk'} + 1} \right)
\]
\[
\hat \phi_{kk'} = \frac{\E_{q}[p_{kk'}](N^Y_{kk'} + \E_{q}[r_{kk'}])}{N^{\Phi}_{\bm{.}kk'} - \E_{q}[p_{kk'}] + 1}
\]
%
Exploiting the conjugacy of the Beta and the negative binomial distributions and assuming that $q(P)=p(P|Y,Z)$, one can show that $p_{kk'}$ can be sampled from a Beta distribution:
%
\begin{equation} \label{eq:pk_update}
p_{kk'} \sim \textrm{Beta}(c\epsilon + N^Y_{kk'}, c(1-\epsilon) + N^\Phi_{kk'}r_{kk'})
\end{equation}
%
so that: $\E_{q}[p_{kk'}] = \frac{c\epsilon + N^Y_{kk'}}{c\epsilon + N^Y_{kk'} + c(1-\epsilon) + N^\Phi_{kk'}r_{kk'}}$.

Lastly, by setting $q(r_{kk'}) = \textrm{Gamma}(a_{kk'},b_{kk'})$, one can show (see supplementary material) that $r_{kk'}$ can be sampled from a Gamma distribution ($r_{kk'} \sim \textrm{Gamma}(r_0c_0+N^Y_{kk'},\frac{1}{c_0  -N^\Phi_{kk'}\log(1-\E_{q}[p_{kk'}])})$ and is such that:
%
\begin{equation}
\E_{q}[r_{kk'}] = \frac{r_0c_0+N^Y_{kk'}}{c_0  -N^\Phi_{kk'}\log(1-\E_{q}[p_{kk'}])}
\end{equation}

\subsection{Stochastic variational inference with stratified sampling}

As mentioned before, we couple CVB with SVI. SVI aims at optimizing ELBO through noisy yet unbiased estimates of its natural gradient \cite{hoffman2013stochastic}. %The noise is associated from the training data that are randomly sampled when updating the variational parameters. 
Suppose  that we sample a dyad $(i,j)$ in the graph $G$ with a distribution $g(i,j)$, then SVI can be summarized as follows:
\begin{enumerate}
\item Maximize local parameter $\gamma_{ij}$ with equation \eqref{eq:maximization},
\item Compute the intermediate global parameter expectation: $\hat N = \frac{\gamma_{ij}}{g(i,j)}$,
\item Update the global parameter: $N^{(t+1)} = (1-\rho_t)N^{(t)} + \rho_t \hat N$.
\end{enumerate}

One bottleneck of the CVB updates is that it computes the expectation on the distribution $q(Z^{-ij})$. Consequently, the maximization step needs to keep in memory the $(\gamma_{ij})$ matrix of size $N^2\times K^2$ which is not convenient for large networks. Thus, following \cite{foulds2013stochastic}, we consider that for large count, the $ij^{th}$ term has a negligible impact on the overall sum and so we omit it. In fact, Foulds et al. shown that this approximation is equivalent to make a online EM MAP estimation.

An advantage of SVI is that it let the freedom to choose a sampling strategy. Stratified sampling was proposed in \cite{gopalan2013efficient} with a SVI inference for the binary MMSB model. We adapt this sampling strategy for our SCVI algorithm. Instead of sampling on a distribution $g(i,j)$ on edges, it samples on a subset $S$ of edges (i.e minibatches) with distribution $h(S)$. For each nodes, we build a link-set $S_1$ containing all its edges, and a non-link set $S_0$ of all its non-links that we scattered in $m$ subsets of equal size such that $S_0 = \cup_{n=1}^m S_{0,n}$. Then for each iteration, we randomly pick one node and either its set $S_1$ or one of its $m$ subset in $S_0$ with
\begin{align*}
h(S)=\begin{cases}
    \frac{1}{\chi N}  & \textrm{ if } S = S_1 \\
    \frac{1}{\chi N m}  & \textrm{ if } S \in S_0 
    \end{cases}
\end{align*}
Where $\chi$ is a factor to take into account the symmetry of the adjacency matrix which is fixed to 1 for directed graph and fixed to 2 otherwise. The sampling method allows to adapt the inference to the sparsity of the graph. Moreover, empirical results show that stratified sampling improves speed and quality of the convergence compared to other sampling strategy \cite{gopalan2013efficient}\cite{kim2013efficient}. In our experiments, we find out that sampling the minibatches in $S_0$ with replacement of the non-links improved the performance for both MMSB and WMMSB.
%Our sampling strategy differ from \cite{gopalan2013efficient}, in the distribution over the set $S_0$, since we sample the subset of non-links with replacement the two distribution differ of a factor $N$.

\textbf{Robbins-Monro Condition} The convergence of the SVI is guaranteed under the Robbin-Monro condition \cite{robbins1951stochastic} that imposes constraints on the gradient step, $\sum \rho_t = \infty$ and $\sum \rho_t^2 < \infty$ which can be obtained with $\rho_t = \frac{1}{(\tau +t)^\kappa}$ with $\kappa \in (0.5, 1]$. Thus, we maintain a gradient step for each of the global parameters $\rho^\Phi$ and $\rho^Y$ accounting respectively for  $N^\Phi$ and $N^Y$. For $N^\Theta$, we maintain individual gradient steps $\rho_i^{\Phi}$ for $1\leq i\leq N$, following \cite{miller2009nonparametric}; this improved both convergence and prediction performance.% Thus for each minibatch we only increase the gradient steps of the nodes that are observed. 
Furthermore, to increase the speed of the inference, we update the global parameter $N^\Phi$ and $N^Y$ only after a minibatch round. For the global parameter $N^\Theta$, we update it after a burn-in period $T_{burnin}$ such that $T_{burnin} \leq |S|$.
%, which consists to update the class assignment of nodes after observing a bunch of dyad.
This heuristic provides a trade-off between updating the global parameters after each observation, which slows down the inference and may result in bad local optima, and updating them only after minibatches that are potentially large (proportional to the number of nodes).

%%% Two more point exists:
% * the time step is increased with the minibatch size,
% * the intermiatade graidnet is normalized by the minibatch size.

%Our SCVI algorithm is summarized in pseudo-code \ref{algo:scvb}.
%
%\begin{algorithm}
%\KwIn{Random initialization of $N^\Theta, N^\Phi, N^Y$.}
%\KwOut{$\Thetah, \Phih$.}
%\Begin{
%$t \gets 0$ \\
%\While{Convergence criteria not met}{
%    Sample a minibatch $S_t$ from $h(S)$. \\
%    \ForEach{$i,j \in S_t$}{
%        Maximize local parameters $\gamma_{ij}$ from \eqref{eq:maximization}.\\
%        \If{burn-in finished}{
%            Compute intermediate gradient $\hat N^\Theta$ from \eqref{eq:sss}.\\
%            Update global parameter $N^{\Theta (t)}$.\\
%            Update gradient step $\rho^\Theta_t$.\\
%        }
%    }
%
%    Compute intermediate gradient $\hat N^\Phi$ and $\hat N^Y$  from \eqref{eq:sss}.\\
%    Update global parameters $N^{\Phi (t)}$ and $N^{Y (t)}$.\\
%    Update gradient steps $\rho^\Phi_t$ and $\rho^\Theta_t$.\\
%	Sample $P$ and $R$ from \eqref{eq:pk_update} \eqref{eq:rk_update}.\\
%    $t \gets t + 1$ .
%    }
%}
%\caption{SCVI pseudo-code.}
%\label{algo:scvb}
%\end{algorithm}


\section{Experimental validation}
\label{sec:exps}

We evaluated the performance of the above models on several real world weighted networks, both directed and undirected. Theirs characteristics and properties are summarized in Table \ref{table:corpus} and detailed descriptions are available in the online Koblenz network collection\footnote{http://konect.uni-koblenz.de/networks/}. For both astro-ph and hep-ph datasets, we used the cleaned versions available in the  graph-tool framework.
%The aim of these experiments is to illustrate the advantage of the online inference and to evaluate the performances of the models.
%This evaluation is based on a missing weight prediction task using the MSE score. 
 For all the datasets, we built a test set by extracting randomly 20 percent of the edges of the network and about the same amount of non-linked pairs of nodes. The remaining data constitutes the training set. We repeated this sampling 10 times with different seeds to cross validate our results. The average values (and standard deviations) computed on the ten sets are reported as final results.

\begin{table*}[t]
\bgroup
\def\arraystretch{1} % 1 is the default, change whatever you need
	
\caption{Datasets networks used to train the models. Type A is for co-authorship, type C is for communication, type H is for hyperlinks and type L is for lexical network.}

\begin{tabular}{lrrrrcrrrr}
%\Xhline{2\arrayrulewidth}
\toprule
 Datasets     &   Nodes &   Edges &   Density & Directed  &    Diameter &   \multicolumn{3}{c}{Weights}  	& type     \\
 \cmidrule(l){7-9}  &   &   	  &   		  & 		  &  		   	&  mean & std  & max             \\
%\hline
\midrule
 astro-ph      &   16706 &  121251 &     0.001    & False    &       14 &   1.8& 3.3 & 306      & A  \\
 cond-mat      &   16726 &   47594 &     0.000    & False    &       18 &   3.1& 7.2& 544      & A  \\
 hep-th        &    8361 &   15751 &     0.000    & False    &        1 &   5.2& 16& 1226      & A  \\
 netscience    &    1589 &    2742 &     0.002    & False    &        2 &   2.2& 1.9 &33       & A  \\
 manufacturing &     167 &    5783 &     0.209    & True     &        3 &  14.3& 44.9&1458    & C  \\
 fb\_uc        &    1899 &   20296 &     0.006    & True     &        4 &   2.8& 4.7& 98       & C  \\
 enron         &    87273  & 320154   &  0.0000   & True     &       15 &   3.4& 12.4&3904    & C  \\
 wiki-link     &  100312   & 887426   & 0.0001    & True     &       14 &   1.7& 3.0& 185      & H  \\
%\Xhline{2\arrayrulewidth}
\bottomrule
\end{tabular}


\egroup
\label{table:corpus}
\end{table*}

In the remainder, for the MMSB, WMMSB and WMMSB-bg models, the gradient step parameters  $\tau$ and $\kappa$ were fixed respectively to  $1024$ and $0.5$, the burn-in period $T_{burnin}$ to $150$; for stratified sampling, $M$ was set to $50$, the size of $s_0^{i,m}, \, 1 \le m \le M$ being equal to the number of nodes to which $i$ is not connected to divided by $M$. For MMSB, the hyperparameters $\lambda_0$ and $\lambda_1$ were set to $0.1$. For WMMSB, $r$ et $p$ were set to $1$ ??? and for WMMSB-bg the hyperparameters were set to  $c_0=10$, $r_0=1$, $c=100$ and $\epsilon=10^{-6}$. The number of latent classes $K$ was fixed to $10$ for all models and the latent-class hyperparameters $\alpha_k$ to $\frac{1}{K}$. The implementation of these models is available online\footnote{https://github.com/***/*** (anonymized)}. For deciding when to stop the inference process, 10\% of the training set used serves as a validation set on which the log-likelihood is computed after each minibatch iteration. When the increase of the log-likelihood, averaged over the last 20 measures, is less than 0.001, the inference is stopped. The log-likelihood of a given set of observations $\D_{set}$  is given by:
%
\begin{equation*}
\log p(\D_{set}) = \sum_{i,j \in \D_{set}} \log p(y_{ij} | \phih_{kk'}) p(k|\thetah_i) p(k'|\thetah_j).
%\log p(\D_{test}) = \sum_{i,j \in \D_{test}} \log p(y_{ij} | \phih_{kk'}) p(k|\thetah_i) p(k'|\thetah_j)
\end{equation*}
%

Predicting links and predicting weights on links are too different tasks, and there is no guarantee that a model performing well on one task will perform well on the other. We nevertheless assess the behavior of the weighted mixed-membership model we have introduced in the context of these two tasks to fully illustrate its performance, however giving more emphasis to weight prediction.

\subsection{Link prediction}

We want to illustrate here how the MMSB and WMMSB-bg models behave for link prediction. In addition to these models, we consider here two standard link prediction models, the stochastic block model, referred to as SBM, and its weighted extension, referred to as WSBM. For these two models, the microcanonical stochastic block model implementation of \cite{peixoto2018nonparametric} has been used since it integrates an efficient MCMC inference method for the stochastic block model family.  In all models, the number of classes is set to $K=10$. 

As usual, the missing link prediction task is evaluated with the AUC-ROC score. For weighted models, we simply predict here a link through the probability that an edge exists between two unobserved nodes $(i,j)$ belonging to the test set, namely:
\[
p(y_{ij} \geq 1 | \Bs{\Thetah}, \Bs{\Phih}) = 1 - \sum_{kk'} \thetah_{ik} \thetah_{jk'} e^{-\phih_{kk'}}
\]

%Variational inference, used here for MMSB models, and MCMC, used for SBM models, lead to different performance, the latter usually yielding better models than the former \cite{asuncion2009smoothing}. Indeed, despite the fact that the MMSB models considered here rely on more realistic assumptions regarding the distribution of nodes over latent classes, the approximations made on the likelihood for scalable inference purposes penalize MMSB models when it comes to prediction accuracy. This said, the strong averaging step of the stochastic gradient descent allows for faster convergence so that, as the models are more realistic, they may yield better performance when the amount of training data is limited. This is indeed what we observe in practice.

Table~\ref{table:roc} summarizes the results obtained with the above mentioned models when using 10\% and 100\% of the training data. As one can note, using all training data, SBM outperforms WSBM on 5 datasets and is the best performing model when the complete training set is used. This can be attributed to the fact that SBM directly aims at predicting links, unlike the weighted models, and does so via MCMC inference, which is known to yield accurate estimate when there is sufficient data. Interestingly, there is an important degradation for SBM models when only 10\% of the training set is used. MMSB models are more stable in this aspect, showing that the stochastic variational inference used in MMSB models allows one to learn a correct model with few data.

\begin{table*}[t]
\centering
	
\caption{Comparison of MMSB, WMMSB-bg, SBM and WSBM in terms of AUC-ROC when using 10\% and 100\% of the training data.}

%\resizebox{12cm}{!}{
\resizebox{\textwidth}{!}{

\begin{tabular}{lllll|llll}
\toprule
&   \multicolumn{4}{c}{10\%} &   \multicolumn{4}{c}{100\%}   \\
\cmidrule(l){2-5} \cmidrule(l){6-9}   & MMSB & WMMSB-bg & SBM & WSBM & MMSB & WMMSB-bg & SBM & WSBM   \\
%\hline
\midrule
astro-ph      & \textbf{708} $\pm$ 3  & 700 $\pm$ 30           & 594 $\pm$ 16 & 586 $\pm$ 9           & \textbf{716} $\pm$ 11 & 710 $\pm$ 18          & 701 $\pm$ 6           & 705 $\pm$ 5 \\
hep-th        & \textbf{617} $\pm$ 11 & 579 $\pm$ 12           & 480 $\pm$ 9  & 482 $\pm$ 26          & 675 $\pm$ 8           & 676 $\pm$ 8           & \textbf{779} $\pm$ 10 & 714 $\pm$ 7 \\
moreno\_names & 680 $\pm$ 72          & \textbf{707} $\pm$ 29  & 571 $\pm$ 29 & 594 $\pm$ 30          & 738 $\pm$ 33          & 739 $\pm$ 7           & \textbf{862} $\pm$ 7  & 859 $\pm$ 11 \\
fb\_uc        & 732 $\pm$ 127         & \textbf{827} $\pm$ 8   & 726 $\pm$ 20 & 788 $\pm$ 18          & 784 $\pm$ 140         & 850 $\pm$ 20          & \textbf{902} $\pm$ 2  & 896 $\pm$ 2 \\
digg\_reply   & 485 $\pm$ 178         & \textbf{651} $\pm$ 127 & 551 $\pm$ 47 & 582 $\pm$ 35          & 482 $\pm$ 204         & \textbf{744} $\pm$ 15 & 728 $\pm$ 26          & 717 $\pm$ 17 \\
slashdot      & 519 $\pm$ 193         & \textbf{820} $\pm$ 6   & 721 $\pm$ 66 & 732 $\pm$ 81          & 634 $\pm$ 181         & 791 $\pm$ 11          & 830 $\pm$ 16          & \textbf{834} $\pm$ 12  \\
enron         & 459 $\pm$ 289         & 875 $\pm$ 14           & 870 $\pm$ 80 & \textbf{923} $\pm$ 14 & 529 $\pm$ 256         & 835 $\pm$ 8           & 799 $\pm$ 20          & \textbf{853} $\pm$ 63  \\
wiki-link     & 491 $\pm$ 242         & 739 $\pm$ 73           & 848 $\pm$ 4  & \textbf{850} $\pm$ 4  & 432 $\pm$ 185         & 785 $\pm$ 8           & \textbf{925} $\pm$ 2  & 915 $\pm$ 3 \\
prosper-loans & 548 $\pm$ 284         & \textbf{752} $\pm$ 11  & 466 $\pm$ 57 & 455 $\pm$ 44          & 434 $\pm$ 274         & \textbf{727} $\pm$ 30 & 500 $\pm$ 4           & 504 $\pm$ 6 \\
\bottomrule
\end{tabular}
}

% 10 100
%\begin{tabular}{lllll|llll}
%\toprule
%&   \multicolumn{2}{c}{MMSB} &   \multicolumn{2}{c}{WMMSB-bg} &   \multicolumn{2}{c}{SBM} & \multicolumn{2}{c}{WSBM}   \\
%\cmidrule(l){2-3} \cmidrule(l){4-5} \cmidrule(l){6-7}\cmidrule(l){8-9}  & 10 & 100 & 10 & 100 & 10 &  100 & 10 & 100   \\
%%\hline
%\midrule                              
%astro-ph        &  \textbf{708} $\pm$ 3    & \underline{716} $\pm$ 11          & 700 $\pm$ 30            &  710 $\pm$ 18   &   594 $\pm$ 16   &  701 $\pm$ 6               &   588 $\pm$ 12           & 705 $\pm$ 5  \\
%hep-th          &  \textbf{617} $\pm$ 11   & 675 $\pm$ 8           & 579 $\pm$ 12            &  676 $\pm$ 8    &   480 $\pm$ 9    &  \underline{779} $\pm$ 1               &   497 $\pm$ 29            & 716 $\pm$ 9  \\
%moreno\_names   &  680 $\pm$ 72            & 738 $\pm$ 33          & \textbf{707} $\pm$ 29   &  739 $\pm$ 7    &   571 $\pm$ 29   &  \underline{862} $\pm$ 7               &   588 $\pm$ 25            & 862 $\pm$ 10 \\
%fb\_uc          &  732 $\pm$ 127           & 784 $\pm$ 14          & \textbf{827} $\pm$ 8    &  850 $\pm$ 20   &   726 $\pm$ 20   &  \underline{902} $\pm$ 2               &   787 $\pm$ 15            & 896 $\pm$ 2  \\
%digg\_reply     &  485 $\pm$ 178           & 482 $\pm$ 204         & \textbf{651} $\pm$ 127  &  744 $\pm$ 15   &   551 $\pm$ 47   &  728 $\pm$ 26              &   584 $\pm$ 34            & 714 $\pm$ 17 \\
%slashdot        &  519 $\pm$ 193           & 634 $\pm$ 181         & \textbf{820} $\pm$ 6    &  791 $\pm$ 11   &   721 $\pm$ 66   &  830 $\pm$ 16              &   699 $\pm$ 79            & \underline{833} $\pm$ 13 \\
%enron           &  459 $\pm$ 289           & 529 $\pm$ 256         & \textbf{875} $\pm$ 14   &  835 $\pm$ 8    &   870 $\pm$ 80   &  799 $\pm$ 20              &   866 $\pm$ 45            & \underline{842} $\pm$ 51 \\
%wiki-link       &  491 $\pm$ 242           & 432 $\pm$ 185         & 739 $\pm$ 73            &  785 $\pm$ 8    &   848 $\pm$ 4    &  \underline{925} $\pm$ 2               &   \textbf{853} $\pm$ 4    & 914 $\pm$ 4  \\
%prosper-loans   &  548 $\pm$ 284           & 434 $\pm$ 274         & \textbf{752} $\pm$ 11   &  \underline{727} $\pm$ 30   &   466 $\pm$ 57   &  500 $\pm$ 4               &   455 $\pm$ 44  	       & 505 $\pm$ 5  \\
%
%\bottomrule
%\end{tabular}
          
          
          





% 5 20 100

%&   \multicolumn{3}{c}{MMSB} &   \multicolumn{3}{c}{WMMSB-bg} &   \multicolumn{3}{c}{SBM} & \multicolumn{3}{c}{WSBM}     \\
%\cmidrule(l){2-4} \cmidrule(l){5-7} \cmidrule(l){8-10}\cmidrule(l){11-13}  & 5 & 20 & 100 & 5 & 20 & 100 & 5 & 20 & 100 & 5 & 20 & 100   \\

%astro-ph        &  686 $\pm$ 7    & 720 $\pm$ 8    & 716 $\pm$ 11   & 684 $\pm$ 25 & 690 $\pm$ 40  &  710 $\pm$ 18   &    505 $\pm$ 9  &  627 $\pm$ 5   &  701 $\pm$ 6   &  538 $\pm$ 31  &    626 $\pm$ 9   & 705 $\pm$ 5  \\
%hep-th          &  583 $\pm$ 13   & 655 $\pm$ 6    & 675 $\pm$ 8    & 558 $\pm$ 9  & 630 $\pm$ 21  &  676 $\pm$ 8    &   498 $\pm$ 6   &  506 $\pm$ 13  &  779 $\pm$ 1   &  513 $\pm$ 24   &    545 $\pm$ 29  & 716 $\pm$ 9  \\
%moreno\_names   &  674 $\pm$ 43   & 740 $\pm$ 18   & 738 $\pm$ 33   & 664 $\pm$ 39 & 697 $\pm$ 33  &  739 $\pm$ 7    &   478 $\pm$ 59  &  698 $\pm$ 25  &  862 $\pm$ 7   &  457 $\pm$ 38   &    709 $\pm$ 9   & 862 $\pm$ 10 \\
%fb\_uc          &  723 $\pm$ 109  & 728 $\pm$ 140  & 784 $\pm$ 14   & 790 $\pm$ 20 & 846 $\pm$ 11  &  850 $\pm$ 20   &   590 $\pm$ 43  &  846 $\pm$ 13  &  902 $\pm$ 2   &  679 $\pm$ 27   &    855 $\pm$ 7   & 896 $\pm$ 2  \\
%digg\_reply     &  491 $\pm$ 150  & 486 $\pm$ 195  & 482 $\pm$ 204  & 667 $\pm$ 73 & 723 $\pm$ 31  &  744 $\pm$ 15   &   528 $\pm$ 51  &  677 $\pm$ 13  &  728 $\pm$ 26  &  554 $\pm$ 22   &    659 $\pm$ 26  & 714 $\pm$ 17 \\
%slashdot        &  537 $\pm$ 177  & 538 $\pm$ 186  & 634 $\pm$ 181  & 801 $\pm$ 13 & 822 $\pm$ 6   &  791 $\pm$ 11   &   712 $\pm$ 66  &  775 $\pm$ 72  &  830 $\pm$ 16  &  731 $\pm$ 64   &    733 $\pm$ 28  & 833 $\pm$ 13 \\
%enron           &  555 $\pm$ 300  & 576 $\pm$ 301  & 529 $\pm$ 256  & 862 $\pm$ 15 & 876 $\pm$ 9   &  835 $\pm$ 8    &   900 $\pm$ 3   &  898 $\pm$ 39  &  799 $\pm$ 20  &  887 $\pm$ 48   &    916 $\pm$ 29  & 842 $\pm$ 51 \\
%wiki-link       &  389 $\pm$ 192  & 505 $\pm$ 230  & 432 $\pm$ 185  & 749 $\pm$ 43 & 725 $\pm$ 87  &  785 $\pm$ 8    &   517 $\pm$ 59  &  870 $\pm$ 4   &  925 $\pm$ 2   &  592 $\pm$ 159  &    871 $\pm$ 2   & 914 $\pm$ 4  \\
%prosper-loans   &  569 $\pm$ 300  & 622 $\pm$ 233  & 434 $\pm$ 274  & 736 $\pm$ 54 & 746 $\pm$ 28  &  727 $\pm$ 30   &   439 $\pm$ 97  &  503 $\pm$ 74  &  500 $\pm$ 4   &  472 $\pm$ 59   &    504 $\pm$ 39  & 505 $\pm$ 5  \\

                     
                     
                     
                     
                     
                     
                     
                     
                     
               
               
                                                                                                                                        
                     
                                                                                                                                           
                     
                     
                                                                                                                                           
                     
                     
                     
                                                                                                                     









\label{table:roc}
\end{table*}

\subsection{Weight prediction}

The weight prediction task consists in predicting weights on existing links, \textit{i.e.} links for which one knows that $y_{ij} > 0$. For this task, in addition to the previous models, we consider three other stochastic block models, among which two are weighted, from \cite{aicher2014learning} which use a generic variational inference scheme with several kernels: a Bernoulli kernel for the model referred to as SBM-ai, a Normal kernel for the model referred to as WSBM-ai-n and a Poisson kernel for the model referred to as WSBM-ai-p. Lastly, we also consider the Edge Partition Model (EPM) proposed in~\cite{zhou2015} (see Section~\ref{sec:relwork}, the inference of which relies on MCMC.

For both WMMSB-bg and EPM, we used the inferred posterior distribution to estimate the missing weights by:
%
\[
\hat y_{ij} | \Bs{\Thetah}, \Bs{\Phih} = \sum_{kk'} \thetah_{ik} \thetah_{jk'} \phih_{kk'}
\]
%
Note that the Zero Truncated Poisson version of WMMSB-bg would lead to:
%
\[
\hat y_{ij} | \Bs{\Thetah}, \Bs{\Phih} = \sum_{kk'} \thetah_{ik} \thetah_{jk'} \frac{\phih_{kk'}}{(1 - \exp(-\phih_{kk'}))}
\]
%
In practice however, even though the latter version is more appropriate, we have not seen any significant difference between the two versions and present only the results obtained with the former so as to be on the same setting as the other models.

Since the stochastic block models have been primarily designed for solving the link prediction task, we do not have a posterior distribution adapted for weight prediction. Therefore, we used a post estimation of the average weight value in each interaction based on the observed data. More formally, given $O_{kk'}$ the number of observed links and non links between the inferred latent classes $k$ and $k'$, the prediction of the weight on the link between two nodes $i$ and $j$ of class $k$ and $k'$ is given by:
%
\[
\hat y_{ij} | e_i \in \text{class } k, e_j \in \text{class }k' = \sum_{y_{ij} \in O_{kk'}} \frac{y_{ij}}{\# O_{kk'}}.
\]
%
The same method is used for the weighted stochastic block models WSBM, WSBM-ai, WSBM-ai-n and WSBM-ai-p as we observed no significant difference \textcolor{red}{With what?} and even better results using this latter estimation.

\begin{table*}[t]
\centering
	\caption{Comparison of models in terms of MSE on different networks for $K=10$. Best results are in bold.}

\resizebox{\textwidth}{!}{
\begin{tabular}{lllllllll}
\hline
               & MMSB                 & WMMSB-bg                   & EPM                       & SBM-ai              & WSBM-ai-n          & WSBM-ai-p            & SBM               & WSBM            \\
\hline                                                                                                                                                                                      
astro-ph       & \textbf{16.884} $\pm$ 0.81 & 20.182 $\pm$ 3.77          & 18.475 $\pm$ 0.71         & 18.448 $\pm$ 0.67   & 18.543 $\pm$ 0.55  & 18.127 $\pm$ 1.0     & 19.133 $\pm$ 0.89    & 18.298 $\pm$ 0.92   \\
enron          & 58.009 $\pm$ 4.33          & \textbf{54.038} $\pm$ 3.6  & 61.337 $\pm$ 4.4          & 63.929 $\pm$ 4.17   & 66.878 $\pm$ 6.59  & 66.450 $\pm$ 6.6     & 58.384 $\pm$ 4.14    & 64.306 $\pm$ 5.28   \\
fb\_uc         & 15.240 $\pm$ 1.65          & 13.927 $\pm$ 1.06          & \textbf{12.966} $\pm$ 1.08& 15.746 $\pm$ 1.56   & 18.298 $\pm$ 1.51  & 16.812 $\pm$ 1.2     & 15.456 $\pm$ 2.37    & 15.199 $\pm$ 2.39   \\
hep-th         & 111.218 $\pm$ 9.3          & \textbf{94.274} $\pm$ 8.23 & 121.334 $\pm$ 8.01        & 123.217 $\pm$ 14.85 & 122.316 $\pm$ 7.18 & 120.046 $\pm$ 12.53  & 121.755 $\pm$ 10.1   & 108.884 $\pm$ 14.77 \\
wiki-link      & 5.864 $\pm$ 0.38           & \textbf{5.206} $\pm$ 0.16  & 5.633 $\pm$ 0.07          & 5.942 $\pm$ 0.09    & 5.874 $\pm$ 0.25   & 6.014 $\pm$ 0.14     & 5.993 $\pm$ 0.1      & 5.936 $\pm$ 0.2     \\
moreno\_names  & 5.334 $\pm$ 1.11           & 4.521 $\pm$ 0.89           & \textbf{3.077} $\pm$ 0.5  & 6.314 $\pm$ 1.77    & 6.355 $\pm$ 1.44   & 5.552 $\pm$ 0.9      & 4.616 $\pm$ 1.07     & 4.809 $\pm$ 1.48    \\
digg-reply     & 1.528 $\pm$ 0.3            & \textbf{0.956} $\pm$ 0.04  & 2.022 $\pm$ 0.01          & 2.026 $\pm$ 0.0     & 2.028 $\pm$ 0.0    & 2.026 $\pm$ 0.0      & 2.022 $\pm$ 0.01     & 2.025 $\pm$ 0.01    \\
prosper-loans  & 1.590 $\pm$ 0.43           & \textbf{1.217} $\pm$ 0.12  & 1.944 $\pm$ 0.0           & 2.043 $\pm$ 0.0     & 2.048 $\pm$ 0.0    & 2.043 $\pm$ 0.0      & 1.981 $\pm$ 0.0      & 1.979 $\pm$ 0.0     \\
slashdot       & 1.459 $\pm$ 0.19           & \textbf{0.907} $\pm$ 0.04  & 2.124 $\pm$ 0.01          & 2.164 $\pm$ 0.01    & 2.167 $\pm$ 0.0    & 2.173 $\pm$ 0.01     & 2.138 $\pm$ 0.0      & 2.144 $\pm$ 0.01    \\
\hline
\end{tabular}

}


\label{table:mse}
\end{table*}

Table \ref{table:mse}, gives the MSE scores for the different models and for all networks. To evaluate the ability of the models to reconstruct the missing weights, we compute the MSE only on the missing edges of the test data. Although the models are also able to predict the presence or absence of edges,  we considered that the question of recovering the edges structure should be addressed separately by a link prediction method. 
As one can note, WMMSB-bg outperforms all the other models on the different datasets except for the network moreno\_names. 
The fact that this network is the only one in the category of linguistic network and that it is relatively small could explain this result. \textbf{CL : je ne suis pas d'accord avec phrase suivante} EPM outperforms other models, apart from WMMSB-bg. The other models expose comparable results.


\begin{figure}[h]
\centering
	
\begin{subfigure}
     \centering
         \includegraphics[width=0.23\textwidth]{fig2/astro-ph_wsim_evo2__}
\end{subfigure}
\begin{subfigure}
         \centering
      \includegraphics[width=0.23\textwidth]{fig2/digg-reply_wsim_evo2__}               
\end{subfigure}                                                                          
\begin{subfigure}                                                                        
         \centering                                                                      
      \includegraphics[width=0.23\textwidth]{fig2/fb_uc_wsim_evo2__}
\end{subfigure}                                                                          
\begin{subfigure}                                                                        
         \centering                                                                      
      \includegraphics[width=0.23\textwidth]{fig2/hep-th_wsim_evo2__}
\end{subfigure}                                                                          
\begin{subfigure}
         \centering
      \includegraphics[width=0.23\textwidth]{fig2/prosper-loans_wsim_evo2__}
\end{subfigure}                                                             
\begin{subfigure}                                                           
         \centering                                                         
      \includegraphics[width=0.23\textwidth]{fig2/slashdot_wsim_evo2__}
\end{subfigure}                                                             
\caption{Performance sensibility when the number of latent classes vary from k=10 to k=50.}

   \label{fig:k_evolv}
\end{figure}
Figure \ref{fig:k_evolv} shows the MSE score for $K$, the number of latent classes, varying from 10 to 50 and for six datasets. One can see that for the different methods, the result is relatively stable, in a lesser extent on the networks hep\_th and fb\_uc. Thus, the choice of this parameter has a limited impact on the performances of the models. Nevertheless, it should be notice that for $K$ equals  to 50, EPM has not been able to handle the datasets.

\textbf{CL: Pbl axe 2 indique Wsim et non MSE}

%%%%%%%%%%%%%%%%%%%%%%%%%%%%%%%%%%%%%%%%%%%%%%%%%%%%%%%%
%%%% Timing performance
%%%%%%%%%%%%%%%%%%%%%%%%%%%%%%%%%%%%%%%%%%%%%%%%%%%%%%%%
\begin{table*}[t]
\centering
	\caption{Comparison of models in terms of the inference time, in hour, on different training sets for $K=10$.}

\resizebox{\textwidth}{!}{
\begin{tabular}{lllllllll}
\hline
              & MMSB-scvb          & WMMSB-bg           & EPM                          & SBM-ai               & WSBM-ai-n        & WSBM-ai-p          & SBM-gt                  & WSBM-gt                  \\
\hline                                                                                                                                                                  
astro-ph      & 0.09 $\pm$ 0.02    & 0.07 $\pm$ 0.02    & \textbf{1.08}  $\pm$ 0.0     & 0.08 $\pm$ 0.01      & 0.07 $\pm$ 0.01  & 0.08 $\pm$ 0.02    & 0.01 $\pm$ 0.0          & 0.03 $\pm$ 0.01          \\
enron         & 1.52 $\pm$ 0.62    & 1.29 $\pm$ 0.61    & \textbf{25.03} $\pm$ 0.03    & 1.13 $\pm$ 0.08      & 1.04 $\pm$ 0.05  & 1.00 $\pm$ 0.16    & 0.10 $\pm$ 0.01         & 0.22 $\pm$ 0.01          \\
fb\_uc         & 0.01 $\pm$ 0.0    & 0.02 $\pm$ 0.01    & \textbf{0.02}  $\pm$ 0.0     & 0.01 $\pm$ 0.0       & 0.01 $\pm$ 0.01  & 0.01 $\pm$ 0.0     & 0.00 $\pm$ 0.0          & 0.01 $\pm$ 0.0           \\
hep-th        & 0.05 $\pm$ 0.01    & 0.08 $\pm$ 0.04    & \textbf{0.26}  $\pm$ 0.01    & 0.02 $\pm$ 0.0       & 0.02 $\pm$ 0.01  & 0.02 $\pm$ 0.0     & 0.00 $\pm$ 0.0          & 0.01 $\pm$ 0.0           \\
wiki-link     & 2.15 $\pm$ 0.5     & 1.67 $\pm$ 0.34    & \textbf{25.10} $\pm$ 0.03    & 1.38 $\pm$ 0.16      & 1.48 $\pm$ 0.08  & 1.58 $\pm$ 0.33    & 0.29 $\pm$ 0.01         & 0.63 $\pm$ 0.05          \\
moreno\_names  & 0.02 $\pm$ 0.02   & 0.01 $\pm$ 0.01    & \textbf{0.02}  $\pm$ 0.0     & 0.01 $\pm$ 0.0       & 0.01 $\pm$ 0.01  & 0.01 $\pm$ 0.0     & 0.01 $\pm$ 0.0          & 0.01 $\pm$ 0.0           \\
digg-reply    & 0.51 $\pm$ 0.3     & 0.71 $\pm$ 0.27    & \textbf{4.69}  $\pm$ 0.04    & 0.15 $\pm$ 0.01      & 0.14 $\pm$ 0.01  & 0.15 $\pm$ 0.01    & 0.02 $\pm$ 0.01         & 0.06 $\pm$ 0.01          \\
prosper-loans & 0.87 $\pm$ 0.24    & 1.71 $\pm$ 0.78    & \textbf{25.13} $\pm$ 0.03    & 1.64 $\pm$ 0.09      & 1.62 $\pm$ 0.1   & 1.64 $\pm$ 0.2     & 1.38 $\pm$ 0.09         & 3.04 $\pm$ 0.22          \\
slashdot      & 1.27 $\pm$ 0.24    & 1.51 $\pm$ 0.48    & \textbf{12.98} $\pm$ 1.26    & 0.39 $\pm$ 0.03      & 0.36 $\pm$ 0.03  & 0.39 $\pm$ 0.04    & 0.04 $\pm$ 0.01         & 0.08 $\pm$ 0.01          \\
\hline
\end{tabular}
}


\label{table:time}
\end{table*}
Table \ref{table:time} presents the inference time of the models on the different dataset in hours. The models MMSB-scvb and WMMSB-gt are implemented in Python, SBM-ai-b/WSBM-ai-n/WSBM-ai-p in Matlab and SBM-gt and WSBM-gt in C++ (with a Python wrapper \cite{peixoto_graph-tool_2014}), which must be considered when comparing the timing. Furthermore, a hard limit  was set at 25 hours for the inference time to limit the computing cost. We can observe that SBM-gt is the fastest model while EPM is the slowest. This last one can take more than 20 times the inference duration of MMSB and WMMSB; which confirms its incapacity to handle the networks with $K$ equals 50 as seen previously. Notably, the inference of MMSB and WMMSB with the SCVB inference scheme fits millions of edges in less than 2 hours, while it is implemented in Python, which is very satisfying and allows to expect better performances  with optimized implementations for this inference scheme.



%%%%%%%%%%%%%%%%%%%%%%%%%%%%%%%%%%%%%%%%%%%%%%%%%%%%%%%%
%%%% Convergence of *MMSB SCVB inference
%%%%%%%%%%%%%%%%%%%%%%%%%%%%%%%%%%%%%%%%%%%%%%%%%%%%%%%%
Figure \ref{fig:conv_entropy} shows the evolution of the log-likelihood for the MMSB-based models on the test set for enron, slashdot and proper-loans datasets. We used three different sets for the hyperparameters shape $r$ and scale $p$ of WMMSB. Regardless of the values of these hyperparameters, one can observe that the augmented model WMMSB-bg is less prone to overfitting, usually converges to a better solution and only needs a small proportion of the total number $N^2$ of edges to do so.
\textbf{CL : detail sur les 3 ensembles a preciser ?} 

\begin{figure*}[h]
\centering
	%

\begin{subfigure}
     \centering
         \includegraphics[width=0.23\textwidth]{fig/astro-ph_fig__entropy}
\end{subfigure}
\begin{subfigure}
         \centering
      \includegraphics[width=0.23\textwidth]{fig/digg-reply_fig__entropy}               
\end{subfigure}                                                                          
\begin{subfigure}                                                                        
         \centering                                                                      
      \includegraphics[width=0.23\textwidth]{fig/hep-th_fig__entropy}
\end{subfigure}                                                                          
\begin{subfigure}                                                                        
     \centering                                                                          
         \includegraphics[width=0.23\textwidth]{fig/enron_fig__entropy}
\end{subfigure}
\begin{subfigure}
         \centering
      \includegraphics[width=0.23\textwidth]{fig/prosper-loans_fig__entropy}
\end{subfigure}                                                             
\begin{subfigure}                                                           
         \centering                                                         
      \includegraphics[width=0.23\textwidth]{fig/slashdot_fig__entropy}
\end{subfigure}                                                             
\caption{Log-likehood convergence for WMMSB and WMMSB-bg models on a test set containing 20\% of the edges of the networks. Three different sets of hyperparmeters are used for WMMSB.}


	\begin{subfigure}                                                                        
     \centering                                                                          
         \includegraphics[width=0.32\textwidth]{fig/enron_fig__entropy__}
\end{subfigure}
\begin{subfigure}
         \centering
      \includegraphics[width=0.32\textwidth]{fig/prosper-loans_fig__entropy__}
\end{subfigure}                                                             
\begin{subfigure}                                                           
         \centering                                                         
      \includegraphics[width=0.32\textwidth]{fig/slashdot_fig__entropy__}
\end{subfigure}                                                             
\caption{Log-likehood convergence for WMMSB and WMMSB-bg models on a test set containing 20\% of the edges of the networks. Three different sets of hyperparmeters are used for WMMSB.}


    \label{fig:conv_entropy}
\end{figure*}








\section{Conclusion}
\label{sec:concl}

We studied in this paper the problem of modeling weighted networks through generalized stochastic block models. The stochastic block models proposed so far for weighted networks suffer from the same drawback as standard stochastic block models, namely the fact that a node can belong to only one class, which is not realistic for many networks and can be corrected by using mixed-membership block models. We have thus developed new mixed-membership stochastic block models to model (directed or undirected) weighted networks and have proposed a scalable inference method, based on a combination of collapsed and stochastic variational inference. This allowed us to deploy the new models on large networks comprising millions of edges. Experiments conducted on nine real-world networks of different types and sizes showed that the new models outperform previously proposed models on the weight prediction task, with reasonable inference time.

In the future, we want to develop versions of these models with different kernels so as to model signed networks and be able to generate different types of weights, thus extending the set of tools available for network analysis.


\clearpage
%\bibliographystyle{unsrt}
%\bibliographystyle{apalike}
\bibliographystyle{alpha}
%\bibliographystyle{splncs04}
\bibliography{./a}

\clearpage
\appendix




\section{Collapsed Variational Updates Derivation}

The derivation of the Collapsed Variational update is obtained by maximizing the ELBO w.r.t $\gamma_{ijkk'}$ with 

\begin{align*}
\frac{\partial \L}{\partial \gamma_{ijkk'}} &= \frac{\partial }{\partial \gamma_{ijkk'}}  \sum_{Z^{-ij}}\sum_{k_1=1}^K\sum_{k_2=1}^K  q(Z^{-ij}) \gamma_{ijk_1 k_2} (\log p(Y, Z^{-ij}, z_{i\rightarrow j}=k_1, z_{i\leftarrow j}=k_2|\Omega)+ \\
& \qquad \log q(Z^{-ij}, z_{i\rightarrow j}=k_1, z_{i\leftarrow j}=k_2) )   \\
&= E_{q(Z^{-ij})}[ p(Y, Z^{-ij}, z_{i\rightarrow j}=k, z_{i\leftarrow j}=k'|\Omega))] + H[Z^{-ij}] -\log(\gamma_{ijkk'}) +1
\end{align*}

\textcolor{red}{(expliqué la prportionnel.)}

By Equating the derivative to zeros, one obtain the following update
\begin{equation} \label{eq1}
\gamma_{ijkk'} \propto \exp E_{q(Z^{-ij})} [\log P(z_{i\rightarrow j}=k, z_{i\leftarrow j}=k' | Y^{-ij}, Z^{-ij}, \Omega) ]
\end{equation}

with  $P(z_{i\rightarrow j}=k, z_{i\leftarrow j}=k' | Y^{-ij}, Z^{-ij}, \Omega)$ being the collapse Gibbs update in WMMSB with the following close form expression

\begin{align*}
P(z_{i\rightarrow j}=k, z_{i\leftarrow j}=k' |-) \propto (n_{\rightarrow ik}^{\Theta^{-j}} + \alpha_k) (n_{\leftarrow jk}^{\Theta^{-i}} + \alpha_{k'}) \mathrm{NB}\left(y_{ij}; n^{Y^{-ij}}_{kk'} + r, \frac{p}{p\,n^{\Phi^{-ij}}_{\bm{.}kk'} + 1} \right)
\end{align*}

Where the count statistic are obtained from the following equations

%\begin{align} \label{eq:sss}
%    n^{\Theta}_{\rightarrow ik} &= \sum_{j, k'} \gamma_{ijkk'}        & n^{\Theta}_{\leftarrow jk'} &= \sum_{i, k} \gamma_{ijkk'}  \nonumber \\
%    n^{\Phi}_{xkk'} &= \sum_{ij:y_{ij}=x} \gamma_{ijkk'}  & n^{Y}_{kk'} &= \sum_{ij} y_{ij}\gamma_{ijkk'}
%\end{align}

\begin{align*}                                                                                                                                        
&n^{\Theta}_{\rightarrow ik} = \sum_j \delta(\zij=k)\\
&n^{Y}_{kk'} = \sum_{ij} y_{ij}\delta(\zij=k, \zji=k') \\
&n^{\Phi}_{\bm{.}kk'} = \sum_{ij} \delta( \zij=k, \zji=k') 
\end{align*}   

By Applying a first order Taylor expansion on equation \eqref{eq1}, following \cite{teh2006collapsed} one obtain

\begin{equation}
\gamma_{ijkk'} \propto (E_{q(Z^{-ij})}[n_{\rightarrow ik}^{\Theta^{-j}}] + \alpha_k) (E_{q(Z^{-ij})}[n_{\leftarrow jk}^{\Theta^{-i}}] + \alpha_{k'}) \mathrm{NB}\left(y_{ij}; E_{q(Z^{-ij})}[n^{Y^{-ij}}_{kk'}] + r, \frac{p}{p\,E_{q(Z^{-ij})}[n^{\Phi^{-ij}}_{\bm{.}kk'}] + 1} \right)
\end{equation}

Finally, by using a gaussian approximation, one can estimate the expectations $E_{q(Z^{-ij})}[n_{\rightarrow ik}^{\Theta^{-j}}], E_{q(Z^{-ij})}[n_{\leftarrow jk}^{\Theta^{-i}}]$ and  $E_{q(Z^{-ij})}[n^{\Phi^{-ij}}_{\bm{.}kk'}]$ with the counts define in equations (2) (section 3.1).


\section{Beta-Gamma Updates}

In the WMMSB-bg model, we assume the variational distribution to be 

\begin{equation*}
q(\Pi) = q(\Theta, \Phi|Z, R, P) q(Z)q(R)q(P)
\end{equation*}

In our optimization, we set the $q(r_{kk'}) \sim \mathrm{Gamma}(a_{kk'}, b_{kk'})$ iid for $1\leq k \leq K$ and $1\leq k' \leq K$ . 

The collapsed ELBO can be rewritten as 

\begin{align*}
\log p(Y) \geq \L_{Z,R,P} &= \E_{q}[\log p(Y, Z, R, P|\Omega)] + \textrm{H}[q(Z)] + \textrm{H}[q(R)] + \textrm{H}[q(P)] \\
                        &= \E_{q}[\log p(Y, Z)] + \textrm{H}[q(Z)] \\
                        &\qquad + \E_{q}[\log p(R|Y,Z,P)] + \textrm{H}[q(R)] \\
                        &\qquad +\E_{q}[\log p(P|Y,Z)] + \textrm{H}[q(P)] 
\end{align*}

\paragraph{Optimizing $\gamma_{ijkk'}$}

In the beta-gamma augmentation, the parameter $p$ and $r$ are marginalized in the original update equation \eqref{eq1}
\begin{equation}
\gamma_{ijkk'} \propto \exp E_{q(Z^{-ij})} [\log E_{q(r_{kk'})}[E_{q(p_{kk'})}[ P(z_{i\rightarrow j}=k, z_{i\leftarrow j}=k' | Y^{-ij}, Z^{-ij}, \Omega) ] ] ]
\end{equation}

By using a first order Taylor expansion, on obtain the following update

\begin{equation}
\gamma_{ijkk'} \propto (N_{\rightarrow ik}^{\Theta^{-j}} + \alpha_k) (N_{\leftarrow jk}^{\Theta^{-i}} + \alpha_{k'}) \mathrm{NB}\left(y_{ij}; N^{Y^{-ij}}_{kk'} + \E_{q}[r_{kk'}], \frac{\E_{q}[p_{kk'}]}{\E_{q}[p_{kk'}]\,N^{\Phi^{-ij}}_{\bm{.}kk'} + 1} \right)
\end{equation}

\paragraph{Optimizing $r_{kk'}$}

We isolate the part of the ELBO than depends only on $r_{kk'}$ parameters ($a_{kk'}$ and $b_{kk'}$) and we will drop the other terms. Thus, We consider only the links that have been generated within the classes interaction $k,k'$ denoted by $Y^{(kk')}$. Furthermore we know that links in class $k$ are i.i.d drawn from  Negative binomial distribution such that $y_{ij} \sim NB(r_{kk'}, p_{kk'})$ if $(ij)$ is within the classes interaction $k,k'$. One has

\begin{align*}
\L_{[r_{kk'}]} = \E_{q(r_{kk'})}[\log p(r_{kk'}|Y^{(kk')},Z^{(kk')},p_{kk'})] + \textrm{H}[q(r_{kk'})] \\
\end{align*}

By applying a Bayes rules and dropping the normalizing term that do not depend on $r_{kk'}$, on has

\begin{align*}
\L_{[r_{kk'}]} &= \E_{q(r_{kk'})}[\log \left( p(Y^{(kk')}|Z^{(kk')}, r_{kk'}, p_{kk'}) p(r_{kk'}]) \right)] + \textrm{H}[q(r_{kk'})] \\
    &= \E_{q(r_{kk'})}[\log \left( \prod_{ij\in Y^{(kk')}} \dbinom{r_{kk'} + y_{ij}-1}{y_{ij}} (1-p_{kk'})^{r_{kk'}} p_k^{y_{ij}} p(r_{kk'}) \right) ] + \textrm{H}[q(r_{kk'})] \\
    &= \E_{q(r_{kk'})}[\log \left( (1-p_{kk'})^{r_{kk'} N^{\Phi}_{kk'}} p_{kk'}^{N^{Y}_{kk'}} p(r_{kk'}) \prod_{ij\in Y^{(kk')}} \frac{\Gamma(r_{kk'}+y_{ij})}{\Gamma(r_{kk'}) \Gamma(y_{ij}+1) }  \right) ] + \textrm{H}[q(r_{kk'})]
\end{align*}

From the model definitions one has the following identities

\begin{equation*}
\log p(r_{kk'}) = (r_0 c_0-1)\log(r_{kk'}) - r_{kk'} c_0 + \mathrm{cst}
\end{equation*}

and 

\begin{equation*}
\textrm{H}[q(r_{kk'})] = a_{kk'} + \log(b_{kk'}) +\log \Gamma(a_{kk'}) + (1-a_{kk'})\Psi(a_{kk'})
\end{equation*}

If $N^Y_{kk'}=0$, the ELBO takes the following closed form expression

\begin{align*}
\L_{[r_{kk'}]} &= N^\Phi_{kk'} a_{kk'} b_{kk'} \log(1-p_{kk'}) + (r_0 c_0-1 )(\Psi(a_{kk'}) + \log(b_{kk'})) -c_0 a_{kk'} b_{kk'}   \\
&\qquad + a_{kk'} + \log(b_{kk'}) +\log \Gamma(a_{kk'}) + (1-a_{kk'})\Psi(a_{kk'})
\end{align*}

Maximizing with respect to $b_{kk'}$ gives the following update

\begin{equation}\label{eq:update1}
b_{kk'} = \frac{r_0 c_0}{a_{kk'} (c_0 - N^\Phi_{kk'} \log(1-p_{kk'}))}
\end{equation}

If $N^Y_{kk'} \neq 0$, one has $\log \prod_{ij\in Y^{(kk')}} \frac{\Gamma(r_{kk'}+y_{ij})}{\Gamma(r_{kk'}) \Gamma(y_{ij}+1) } = \log \prod_{ij\in Y^{(kk')}} \frac{1}{B(r_{kk'}, y_{ij})y_{ij}}$. From the beta function definition, and assuming that the $y_{ij} \in Y^{(kk')}$ are greater that zeros, one has

\begin{align*}
B(r_{kk'}, y_{ij}) = \int_0^1 t^{r_{kk'}-1} (1-t)^{y_{ij}-1} dt  \leq \int_0^1 t^{r_{kk'}-1} dt = \frac{1}{r_k}
\end{align*}

Therefore plugging the inequality, one has

\begin{equation*}
\log \prod_{ij\in Y^{(kk')}} \frac{\Gamma(r_{kk'}+y_{ij})}{\Gamma(r_{kk'}) \Gamma(y_{ij}+1) } \geq n^Y_{kk'} \log(r_{kk'}) + \mathrm{cst}
\end{equation*}


\textcolor{red}{here, expliqué le passage de $n^Y$ a $N^y$ encore avec une taylor expansion sur q(Z), excatement comme precedement.}
The ELBO can further be lower bounded, which leads to te following form


\begin{align*}
\L_{[r_{kk'}]} &\geq N^\Phi_{kk'} a_{kk'} b_{kk'} \log(1-p_{kk'}) + (r_0 c_0-1 )(\Psi(a_{kk'}) + \log(b_{kk'})) -c_0 a_{kk'} b_{kk'} +N^Y_{kk'} (\Psi(a_{kk'}) + \log(b_{kk'}))  \\
&\qquad a_{kk'} + \log(b_{kk'}) +\log \Gamma(a_{kk'}) + (1-a_{kk'})\Psi(a_{kk'})
\end{align*}

Maximizing again with respect to $b_{kk'}$, the update becomes

\begin{equation} \label{eq:update2}
b_{kk'} = \frac{r_0 c_0 + N^Y_{kk'}}{a_{kk'} (c_0 - N^\Phi_{kk'} \log(1-p_{kk'}))}
\end{equation}

Finally, the maximisation with respect to $a_{kk'}$ leads to update of the form of differential equation involving digamma function. Thus, it has no closed form expression. In the literature, Newtow's method has been suggested to estimate such equation. Instead, we find that $r_{kk'}$ can be recovered from updates \eqref{eq:update1} and \eqref{eq:update2} by remarking that

\begin{equation}
\E_q[r_{kk'}] = a_{kk'} b_{kk'} = \frac{r_0 c_0 + N^Y_{kk'}}{a_{kk'} (c_0 - N^\Phi_{kk'} \log(1-p_{kk'}))}
\end{equation}


One can show that this update is also sound when $N^Y_{kk'} = 0$.


\paragraph{Optimizing $p_{kk'}$}

In oder to maximize the ELBO w.r.t $p_{kk}'$ one can let $q(p_{kk'} =p(p_{kk'} | Y,Z)=E_q(r_{kk'}) [ p(p_{kk'} | Y^{(kk')},Z^{(kk')} ,r_{kk'})]$. As the Negative binomial and beta distribution are conjugate, a close form expression can obtain as

\begin{align*}
p(p_{kk'} | Y^{(kk')}, Z^{(kk')}, r_{kk'}) &\propto p(Y^{(kk')| Z^{(kk')}, r_{kk'}} p(r_{kk'}) \\
                               &\propto (1-p_{kk'})^{r_{kk'} N^\Phi_{kk'}}p_{kk'}^{N^Y_{kk'}} p_{kk'}^{c\epsilon -1} (1-p_{kk'})^{c(1-\epsilon) -1}\\
                               &\propto p_{kk'}^{c\epsilon + N^Y_{kk'} -1} (1-p_{kk'})^{c(1-\epsilon) + N^\Phi_{kk'}r_{kk'}-1}\\
                               &\sim \mathrm{Beta}(c\epsilon + N^Y_{kk'}, c(1-\epsilon) + N^\Phi_{kk'}r_{kk'})
\end{align*}

Finally, by resorting to a first order expansion, one obtain the final update

\begin{equation*}
p(p_{kk'} | Y^{(kk')}, Z^{(kk')}, r_{kk'}) \sim \mathrm{Beta}(c\epsilon + N^Y_{kk'}, c(1-\epsilon) + N^\Phi_{kk'} E_q[r_{kk'}])
\end{equation*}


\section{Stratified Sampling}

Sampling from minibatches in SVI, for MMSB model, was initially proposed in [6] and [7]. The adaptation of the sampling scheme for SCVI is based on the reformulation of the "sufficient statistics" $N^\Theta, N^\Phi$ and $N^Y$  by bringing up a minibatch distribution $h(S)$. The idea of the stratified sampling is to divide the edges into subset that share some statistical strength.
For each node $n$ with divide it's neighbors pairs into a set $S_1$ containing all its links (edges) and a subset $S_0$ dividing into  $m$ set containing its non-links. Then sampling consists of drawing one of its its set $S_0$ or $S_0$ with probability.

\begin{align*}
h(S)=\begin{cases}
    \frac{1}{2 N}  & \textrm{ if } S = S_1 \\
    \frac{1}{2 N m}  & \textrm{ if } S \in S_0 
    \end{cases}
\end{align*}


By referencing any of the global "sufficient statistics" of the models with the term $N^*$ such that $N^* \in \{N^Y, N^\Theta, N^\Phi\}$. Assuming that every pair (i, j) occurs in some constant number of sets c, $N^*$ can be reformulated as follows 

\begin{align*}
N^* = \sum_{ij, *} \gamma_{ij} = \E_h[ \frac{1}{c}\frac{1}{h(S)} \sum_{ij \in S, *} \gamma_{ij}  ]
\end{align*}

Where $N^*$ and $\gamma_{ij}$ are matricies of size $K\times K$.
The exact summing formulation of $N^*=\sum_{ij,*}$ is given in section 3.1. For undirected network, $c$ is equal to 2 because each pair occurs in two set, and $c$ is 1 for directed network.




\end{document}
