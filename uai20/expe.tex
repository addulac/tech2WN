\section{Experimental validation}
\label{sec:exps}

We evaluated the performance of the above models on several real world weighted networks, both directed and undirected. Theirs characteristics and properties are summarized in Table \ref{table:corpus} and detailed descriptions are available in the online Koblenz network collection\footnote{http://konect.uni-koblenz.de/networks/}. For both astro-ph and hep-ph datasets, we used the cleaned versions available in the  graph-tool framework.
%The aim of these experiments is to illustrate the advantage of the online inference and to evaluate the performances of the models.
%This evaluation is based on a missing weight prediction task using the MSE score. 
 For all the datasets, we built a test set by extracting randomly 20 percent of the edges of the network and about the same amount of non-linked pairs of nodes. The remaining data constitutes the training set. We repeated this sampling 10 times with different seeds to cross validate our results. The average values (and standard deviations) computed on the ten sets are reported as final results.

\begin{table*}[t]
\bgroup
\def\arraystretch{1} % 1 is the default, change whatever you need
	
\caption{Datasets networks used to train the models. Type A is for co-authorship, type C is for communication, type H is for hyperlinks and type L is for lexical network.}

\begin{tabular}{lrrrrcrrrr}
%\Xhline{2\arrayrulewidth}
\toprule
 Datasets     &   Nodes &   Edges &   Density & Directed  &    Diameter &   \multicolumn{3}{c}{Weights}  	& type     \\
 \cmidrule(l){7-9}  &   &   	  &   		  & 		  &  		   	&  mean & std  & max             \\
%\hline
\midrule
 astro-ph      &   16706 &  121251 &     0.001    & False    &       14 &   1.8& 3.3 & 306      & A  \\
 cond-mat      &   16726 &   47594 &     0.000    & False    &       18 &   3.1& 7.2& 544      & A  \\
 hep-th        &    8361 &   15751 &     0.000    & False    &        1 &   5.2& 16& 1226      & A  \\
 netscience    &    1589 &    2742 &     0.002    & False    &        2 &   2.2& 1.9 &33       & A  \\
 manufacturing &     167 &    5783 &     0.209    & True     &        3 &  14.3& 44.9&1458    & C  \\
 fb\_uc        &    1899 &   20296 &     0.006    & True     &        4 &   2.8& 4.7& 98       & C  \\
 enron         &    87273  & 320154   &  0.0000   & True     &       15 &   3.4& 12.4&3904    & C  \\
 wiki-link     &  100312   & 887426   & 0.0001    & True     &       14 &   1.7& 3.0& 185      & H  \\
%\Xhline{2\arrayrulewidth}
\bottomrule
\end{tabular}


\egroup
\label{table:corpus}
\end{table*}

In the remainder, for the MMSB, WMMSB and WMMSB-bg models, the gradient step parameters  $\tau$ and $\kappa$ were fixed respectively to  $1024$ and $0.5$, the burn-in period $T_{burnin}$ to $150$; for stratified sampling, $M$ was set to $50$, the size of $s_0^{i,m}, \, 1 \le m \le M$ being equal to the number of nodes to which $i$ is not connected to divided by $M$. For MMSB, the hyperparameters $\lambda_0$ and $\lambda_1$ were set to $0.1$. For WMMSB, $r$ et $p$ were set to $1$ ??? and for WMMSB-bg the hyperparameters were set to  $c_0=10$, $r_0=1$, $c=100$ and $\epsilon=10^{-6}$. The number of latent classes $K$ was fixed to $10$ for all models and the latent-class hyperparameters $\alpha_k$ to $\frac{1}{K}$. The implementation of these models is available online\footnote{https://github.com/***/*** (anonymized)}. For deciding when to stop the inference process, 10\% of the training set used serves as a validation set on which the log-likelihood is computed after each minibatch iteration. When the increase of the log-likelihood, averaged over the last 20 measures, is less than 0.001, the inference is stopped. The log-likelihood of a given set of observations $\D_{set}$  is given by:
%
\begin{equation*}
\log p(\D_{set}) = \sum_{i,j \in \D_{set}} \log p(y_{ij} | \phih_{kk'}) p(k|\thetah_i) p(k'|\thetah_j).
%\log p(\D_{test}) = \sum_{i,j \in \D_{test}} \log p(y_{ij} | \phih_{kk'}) p(k|\thetah_i) p(k'|\thetah_j)
\end{equation*}
%

Predicting links and predicting weights on links are too different tasks, and there is no guarantee that a model performing well on one task will perform well on the other. We nevertheless assess the behavior of the weighted mixed-membership model we have introduced in the context of these two tasks to fully illustrate its performance, however giving more emphasis to weight prediction.

\subsection{Link prediction}

We want to illustrate here how the MMSB and WMMSB-bg models behave for link prediction. In addition to these models, we consider here two standard link prediction models, the stochastic block model, referred to as SBM, and its weighted extension, referred to as WSBM. For these two models, the microcanonical stochastic block model implementation of \cite{peixoto2018nonparametric} has been used since it integrates an efficient MCMC inference method for the stochastic block model family.  In all models, the number of classes is set to $K=10$. 

As usual, the missing link prediction task is evaluated with the AUC-ROC score. For weighted models, we simply predict here a link through the probability that an edge exists between two unobserved nodes $(i,j)$ belonging to the test set, namely:
\[
p(y_{ij} \geq 1 | \Bs{\Thetah}, \Bs{\Phih}) = 1 - \sum_{kk'} \thetah_{ik} \thetah_{jk'} e^{-\phih_{kk'}}
\]

%Variational inference, used here for MMSB models, and MCMC, used for SBM models, lead to different performance, the latter usually yielding better models than the former \cite{asuncion2009smoothing}. Indeed, despite the fact that the MMSB models considered here rely on more realistic assumptions regarding the distribution of nodes over latent classes, the approximations made on the likelihood for scalable inference purposes penalize MMSB models when it comes to prediction accuracy. This said, the strong averaging step of the stochastic gradient descent allows for faster convergence so that, as the models are more realistic, they may yield better performance when the amount of training data is limited. This is indeed what we observe in practice.

Table~\ref{table:roc} summarizes the results obtained with the above mentioned models when using 10\% and 100\% of the training data. As one can note, using all training data, SBM outperforms WSBM on 5 datasets and is the best performing model when the complete training set is used. This can be attributed to the fact that SBM directly aims at predicting links, unlike the weighted models, and does so via MCMC inference, which is known to yield accurate estimate when there is sufficient data. Interestingly, there is an important degradation for SBM models when only 10\% of the training set is used. MMSB models are more stable in this aspect, showing that the stochastic variational inference used in MMSB models allows one to learn a correct model with few data.

\begin{table*}[t]
\centering
	
\caption{Comparison of MMSB, WMMSB-bg, SBM and WSBM in terms of AUC-ROC when using 10\% and 100\% of the training data.}

%\resizebox{12cm}{!}{
\resizebox{\textwidth}{!}{

\begin{tabular}{lllll|llll}
\toprule
&   \multicolumn{4}{c}{10\%} &   \multicolumn{4}{c}{100\%}   \\
\cmidrule(l){2-5} \cmidrule(l){6-9}   & MMSB & WMMSB-bg & SBM & WSBM & MMSB & WMMSB-bg & SBM & WSBM   \\
%\hline
\midrule
astro-ph      & \textbf{708} $\pm$ 3  & 700 $\pm$ 30           & 594 $\pm$ 16 & 586 $\pm$ 9           & \textbf{716} $\pm$ 11 & 710 $\pm$ 18          & 701 $\pm$ 6           & 705 $\pm$ 5 \\
hep-th        & \textbf{617} $\pm$ 11 & 579 $\pm$ 12           & 480 $\pm$ 9  & 482 $\pm$ 26          & 675 $\pm$ 8           & 676 $\pm$ 8           & \textbf{779} $\pm$ 10 & 714 $\pm$ 7 \\
moreno\_names & 680 $\pm$ 72          & \textbf{707} $\pm$ 29  & 571 $\pm$ 29 & 594 $\pm$ 30          & 738 $\pm$ 33          & 739 $\pm$ 7           & \textbf{862} $\pm$ 7  & 859 $\pm$ 11 \\
fb\_uc        & 732 $\pm$ 127         & \textbf{827} $\pm$ 8   & 726 $\pm$ 20 & 788 $\pm$ 18          & 784 $\pm$ 140         & 850 $\pm$ 20          & \textbf{902} $\pm$ 2  & 896 $\pm$ 2 \\
digg\_reply   & 485 $\pm$ 178         & \textbf{651} $\pm$ 127 & 551 $\pm$ 47 & 582 $\pm$ 35          & 482 $\pm$ 204         & \textbf{744} $\pm$ 15 & 728 $\pm$ 26          & 717 $\pm$ 17 \\
slashdot      & 519 $\pm$ 193         & \textbf{820} $\pm$ 6   & 721 $\pm$ 66 & 732 $\pm$ 81          & 634 $\pm$ 181         & 791 $\pm$ 11          & 830 $\pm$ 16          & \textbf{834} $\pm$ 12  \\
enron         & 459 $\pm$ 289         & 875 $\pm$ 14           & 870 $\pm$ 80 & \textbf{923} $\pm$ 14 & 529 $\pm$ 256         & 835 $\pm$ 8           & 799 $\pm$ 20          & \textbf{853} $\pm$ 63  \\
wiki-link     & 491 $\pm$ 242         & 739 $\pm$ 73           & 848 $\pm$ 4  & \textbf{850} $\pm$ 4  & 432 $\pm$ 185         & 785 $\pm$ 8           & \textbf{925} $\pm$ 2  & 915 $\pm$ 3 \\
prosper-loans & 548 $\pm$ 284         & \textbf{752} $\pm$ 11  & 466 $\pm$ 57 & 455 $\pm$ 44          & 434 $\pm$ 274         & \textbf{727} $\pm$ 30 & 500 $\pm$ 4           & 504 $\pm$ 6 \\
\bottomrule
\end{tabular}
}

% 10 100
%\begin{tabular}{lllll|llll}
%\toprule
%&   \multicolumn{2}{c}{MMSB} &   \multicolumn{2}{c}{WMMSB-bg} &   \multicolumn{2}{c}{SBM} & \multicolumn{2}{c}{WSBM}   \\
%\cmidrule(l){2-3} \cmidrule(l){4-5} \cmidrule(l){6-7}\cmidrule(l){8-9}  & 10 & 100 & 10 & 100 & 10 &  100 & 10 & 100   \\
%%\hline
%\midrule                              
%astro-ph        &  \textbf{708} $\pm$ 3    & \underline{716} $\pm$ 11          & 700 $\pm$ 30            &  710 $\pm$ 18   &   594 $\pm$ 16   &  701 $\pm$ 6               &   588 $\pm$ 12           & 705 $\pm$ 5  \\
%hep-th          &  \textbf{617} $\pm$ 11   & 675 $\pm$ 8           & 579 $\pm$ 12            &  676 $\pm$ 8    &   480 $\pm$ 9    &  \underline{779} $\pm$ 1               &   497 $\pm$ 29            & 716 $\pm$ 9  \\
%moreno\_names   &  680 $\pm$ 72            & 738 $\pm$ 33          & \textbf{707} $\pm$ 29   &  739 $\pm$ 7    &   571 $\pm$ 29   &  \underline{862} $\pm$ 7               &   588 $\pm$ 25            & 862 $\pm$ 10 \\
%fb\_uc          &  732 $\pm$ 127           & 784 $\pm$ 14          & \textbf{827} $\pm$ 8    &  850 $\pm$ 20   &   726 $\pm$ 20   &  \underline{902} $\pm$ 2               &   787 $\pm$ 15            & 896 $\pm$ 2  \\
%digg\_reply     &  485 $\pm$ 178           & 482 $\pm$ 204         & \textbf{651} $\pm$ 127  &  744 $\pm$ 15   &   551 $\pm$ 47   &  728 $\pm$ 26              &   584 $\pm$ 34            & 714 $\pm$ 17 \\
%slashdot        &  519 $\pm$ 193           & 634 $\pm$ 181         & \textbf{820} $\pm$ 6    &  791 $\pm$ 11   &   721 $\pm$ 66   &  830 $\pm$ 16              &   699 $\pm$ 79            & \underline{833} $\pm$ 13 \\
%enron           &  459 $\pm$ 289           & 529 $\pm$ 256         & \textbf{875} $\pm$ 14   &  835 $\pm$ 8    &   870 $\pm$ 80   &  799 $\pm$ 20              &   866 $\pm$ 45            & \underline{842} $\pm$ 51 \\
%wiki-link       &  491 $\pm$ 242           & 432 $\pm$ 185         & 739 $\pm$ 73            &  785 $\pm$ 8    &   848 $\pm$ 4    &  \underline{925} $\pm$ 2               &   \textbf{853} $\pm$ 4    & 914 $\pm$ 4  \\
%prosper-loans   &  548 $\pm$ 284           & 434 $\pm$ 274         & \textbf{752} $\pm$ 11   &  \underline{727} $\pm$ 30   &   466 $\pm$ 57   &  500 $\pm$ 4               &   455 $\pm$ 44  	       & 505 $\pm$ 5  \\
%
%\bottomrule
%\end{tabular}
          
          
          





% 5 20 100

%&   \multicolumn{3}{c}{MMSB} &   \multicolumn{3}{c}{WMMSB-bg} &   \multicolumn{3}{c}{SBM} & \multicolumn{3}{c}{WSBM}     \\
%\cmidrule(l){2-4} \cmidrule(l){5-7} \cmidrule(l){8-10}\cmidrule(l){11-13}  & 5 & 20 & 100 & 5 & 20 & 100 & 5 & 20 & 100 & 5 & 20 & 100   \\

%astro-ph        &  686 $\pm$ 7    & 720 $\pm$ 8    & 716 $\pm$ 11   & 684 $\pm$ 25 & 690 $\pm$ 40  &  710 $\pm$ 18   &    505 $\pm$ 9  &  627 $\pm$ 5   &  701 $\pm$ 6   &  538 $\pm$ 31  &    626 $\pm$ 9   & 705 $\pm$ 5  \\
%hep-th          &  583 $\pm$ 13   & 655 $\pm$ 6    & 675 $\pm$ 8    & 558 $\pm$ 9  & 630 $\pm$ 21  &  676 $\pm$ 8    &   498 $\pm$ 6   &  506 $\pm$ 13  &  779 $\pm$ 1   &  513 $\pm$ 24   &    545 $\pm$ 29  & 716 $\pm$ 9  \\
%moreno\_names   &  674 $\pm$ 43   & 740 $\pm$ 18   & 738 $\pm$ 33   & 664 $\pm$ 39 & 697 $\pm$ 33  &  739 $\pm$ 7    &   478 $\pm$ 59  &  698 $\pm$ 25  &  862 $\pm$ 7   &  457 $\pm$ 38   &    709 $\pm$ 9   & 862 $\pm$ 10 \\
%fb\_uc          &  723 $\pm$ 109  & 728 $\pm$ 140  & 784 $\pm$ 14   & 790 $\pm$ 20 & 846 $\pm$ 11  &  850 $\pm$ 20   &   590 $\pm$ 43  &  846 $\pm$ 13  &  902 $\pm$ 2   &  679 $\pm$ 27   &    855 $\pm$ 7   & 896 $\pm$ 2  \\
%digg\_reply     &  491 $\pm$ 150  & 486 $\pm$ 195  & 482 $\pm$ 204  & 667 $\pm$ 73 & 723 $\pm$ 31  &  744 $\pm$ 15   &   528 $\pm$ 51  &  677 $\pm$ 13  &  728 $\pm$ 26  &  554 $\pm$ 22   &    659 $\pm$ 26  & 714 $\pm$ 17 \\
%slashdot        &  537 $\pm$ 177  & 538 $\pm$ 186  & 634 $\pm$ 181  & 801 $\pm$ 13 & 822 $\pm$ 6   &  791 $\pm$ 11   &   712 $\pm$ 66  &  775 $\pm$ 72  &  830 $\pm$ 16  &  731 $\pm$ 64   &    733 $\pm$ 28  & 833 $\pm$ 13 \\
%enron           &  555 $\pm$ 300  & 576 $\pm$ 301  & 529 $\pm$ 256  & 862 $\pm$ 15 & 876 $\pm$ 9   &  835 $\pm$ 8    &   900 $\pm$ 3   &  898 $\pm$ 39  &  799 $\pm$ 20  &  887 $\pm$ 48   &    916 $\pm$ 29  & 842 $\pm$ 51 \\
%wiki-link       &  389 $\pm$ 192  & 505 $\pm$ 230  & 432 $\pm$ 185  & 749 $\pm$ 43 & 725 $\pm$ 87  &  785 $\pm$ 8    &   517 $\pm$ 59  &  870 $\pm$ 4   &  925 $\pm$ 2   &  592 $\pm$ 159  &    871 $\pm$ 2   & 914 $\pm$ 4  \\
%prosper-loans   &  569 $\pm$ 300  & 622 $\pm$ 233  & 434 $\pm$ 274  & 736 $\pm$ 54 & 746 $\pm$ 28  &  727 $\pm$ 30   &   439 $\pm$ 97  &  503 $\pm$ 74  &  500 $\pm$ 4   &  472 $\pm$ 59   &    504 $\pm$ 39  & 505 $\pm$ 5  \\

                     
                     
                     
                     
                     
                     
                     
                     
                     
               
               
                                                                                                                                        
                     
                                                                                                                                           
                     
                     
                                                                                                                                           
                     
                     
                     
                                                                                                                     









\label{table:roc}
\end{table*}

\subsection{Weight prediction}

The weight prediction task consists in predicting weights on existing links, \textit{i.e.} links for which one knows that $y_{ij} > 0$. For this task, in addition to the previous models, we consider three other stochastic block models, among which two are weighted, from \cite{aicher2014learning} which use a generic variational inference scheme with several kernels: a Bernoulli kernel for the model referred to as SBM-ai, a Normal kernel for the model referred to as WSBM-ai-n and a Poisson kernel for the model referred to as WSBM-ai-p. Lastly, we also consider the Edge Partition Model (EPM) proposed in~\cite{zhou2015} (see Section~\ref{sec:relwork}, the inference of which relies on MCMC.

For both WMMSB-bg and EPM, we used the inferred posterior distribution to estimate the missing weights by:
%
\[
\hat y_{ij} | \Bs{\Thetah}, \Bs{\Phih} = \sum_{kk'} \thetah_{ik} \thetah_{jk'} \phih_{kk'}
\]
%
Note that the Zero Truncated Poisson version of WMMSB-bg would lead to:
%
\[
\hat y_{ij} | \Bs{\Thetah}, \Bs{\Phih} = \sum_{kk'} \thetah_{ik} \thetah_{jk'} \frac{\phih_{kk'}}{(1 - \exp(-\phih_{kk'}))}
\]
%
In practice however, even though the latter version is more appropriate, we have not seen any significant difference between the two versions and present only the results obtained with the former so as to be on the same setting as the other models.

Since the stochastic block models have been primarily designed for solving the link prediction task, we do not have a posterior distribution adapted for weight prediction. Therefore, we used a post estimation of the average weight value in each interaction based on the observed data. More formally, given $O_{kk'}$ the number of observed links and non links between the inferred latent classes $k$ and $k'$, the prediction of the weight on the link between two nodes $i$ and $j$ of class $k$ and $k'$ is given by:
%
\[
\hat y_{ij} | e_i \in \text{class } k, e_j \in \text{class }k' = \sum_{y_{ij} \in O_{kk'}} \frac{y_{ij}}{\# O_{kk'}}.
\]
%
The same method is used for the weighted stochastic block models WSBM, WSBM-ai, WSBM-ai-n and WSBM-ai-p as we observed no significant difference \textcolor{red}{With what?} and even better results using this latter estimation.

\begin{table*}[t]
\centering
	\caption{Comparison of models in terms of MSE on different networks for $K=10$. Best results are in bold.}

\resizebox{\textwidth}{!}{
\begin{tabular}{lllllllll}
\hline
               & MMSB                 & WMMSB-bg                   & EPM                       & SBM-ai              & WSBM-ai-n          & WSBM-ai-p            & SBM               & WSBM            \\
\hline                                                                                                                                                                                      
astro-ph       & \textbf{16.884} $\pm$ 0.81 & 20.182 $\pm$ 3.77          & 18.475 $\pm$ 0.71         & 18.448 $\pm$ 0.67   & 18.543 $\pm$ 0.55  & 18.127 $\pm$ 1.0     & 19.133 $\pm$ 0.89    & 18.298 $\pm$ 0.92   \\
enron          & 58.009 $\pm$ 4.33          & \textbf{54.038} $\pm$ 3.6  & 61.337 $\pm$ 4.4          & 63.929 $\pm$ 4.17   & 66.878 $\pm$ 6.59  & 66.450 $\pm$ 6.6     & 58.384 $\pm$ 4.14    & 64.306 $\pm$ 5.28   \\
fb\_uc         & 15.240 $\pm$ 1.65          & 13.927 $\pm$ 1.06          & \textbf{12.966} $\pm$ 1.08& 15.746 $\pm$ 1.56   & 18.298 $\pm$ 1.51  & 16.812 $\pm$ 1.2     & 15.456 $\pm$ 2.37    & 15.199 $\pm$ 2.39   \\
hep-th         & 111.218 $\pm$ 9.3          & \textbf{94.274} $\pm$ 8.23 & 121.334 $\pm$ 8.01        & 123.217 $\pm$ 14.85 & 122.316 $\pm$ 7.18 & 120.046 $\pm$ 12.53  & 121.755 $\pm$ 10.1   & 108.884 $\pm$ 14.77 \\
wiki-link      & 5.864 $\pm$ 0.38           & \textbf{5.206} $\pm$ 0.16  & 5.633 $\pm$ 0.07          & 5.942 $\pm$ 0.09    & 5.874 $\pm$ 0.25   & 6.014 $\pm$ 0.14     & 5.993 $\pm$ 0.1      & 5.936 $\pm$ 0.2     \\
moreno\_names  & 5.334 $\pm$ 1.11           & 4.521 $\pm$ 0.89           & \textbf{3.077} $\pm$ 0.5  & 6.314 $\pm$ 1.77    & 6.355 $\pm$ 1.44   & 5.552 $\pm$ 0.9      & 4.616 $\pm$ 1.07     & 4.809 $\pm$ 1.48    \\
digg-reply     & 1.528 $\pm$ 0.3            & \textbf{0.956} $\pm$ 0.04  & 2.022 $\pm$ 0.01          & 2.026 $\pm$ 0.0     & 2.028 $\pm$ 0.0    & 2.026 $\pm$ 0.0      & 2.022 $\pm$ 0.01     & 2.025 $\pm$ 0.01    \\
prosper-loans  & 1.590 $\pm$ 0.43           & \textbf{1.217} $\pm$ 0.12  & 1.944 $\pm$ 0.0           & 2.043 $\pm$ 0.0     & 2.048 $\pm$ 0.0    & 2.043 $\pm$ 0.0      & 1.981 $\pm$ 0.0      & 1.979 $\pm$ 0.0     \\
slashdot       & 1.459 $\pm$ 0.19           & \textbf{0.907} $\pm$ 0.04  & 2.124 $\pm$ 0.01          & 2.164 $\pm$ 0.01    & 2.167 $\pm$ 0.0    & 2.173 $\pm$ 0.01     & 2.138 $\pm$ 0.0      & 2.144 $\pm$ 0.01    \\
\hline
\end{tabular}

}


\label{table:mse}
\end{table*}

Table \ref{table:mse}, gives the MSE scores for the different models and for all networks. To evaluate the ability of the models to reconstruct the missing weights, we compute the MSE only on the missing edges of the test data. Although the models are also able to predict the presence or absence of edges,  we considered that the question of recovering the edges structure should be addressed separately by a link prediction method. 
As one can note, WMMSB-bg outperforms all the other models on the different datasets except for the network moreno\_names. 
The fact that this network is the only one in the category of linguistic network and that it is relatively small could explain this result. \textbf{CL : je ne suis pas d'accord avec phrase suivante} EPM outperforms other models, apart from WMMSB-bg. The other models expose comparable results.


\begin{figure}[h]
\centering
	
\begin{subfigure}
     \centering
         \includegraphics[width=0.23\textwidth]{fig2/astro-ph_wsim_evo2__}
\end{subfigure}
\begin{subfigure}
         \centering
      \includegraphics[width=0.23\textwidth]{fig2/digg-reply_wsim_evo2__}               
\end{subfigure}                                                                          
\begin{subfigure}                                                                        
         \centering                                                                      
      \includegraphics[width=0.23\textwidth]{fig2/fb_uc_wsim_evo2__}
\end{subfigure}                                                                          
\begin{subfigure}                                                                        
         \centering                                                                      
      \includegraphics[width=0.23\textwidth]{fig2/hep-th_wsim_evo2__}
\end{subfigure}                                                                          
\begin{subfigure}
         \centering
      \includegraphics[width=0.23\textwidth]{fig2/prosper-loans_wsim_evo2__}
\end{subfigure}                                                             
\begin{subfigure}                                                           
         \centering                                                         
      \includegraphics[width=0.23\textwidth]{fig2/slashdot_wsim_evo2__}
\end{subfigure}                                                             
\caption{Performance sensibility when the number of latent classes vary from k=10 to k=50.}

   \label{fig:k_evolv}
\end{figure}
Figure \ref{fig:k_evolv} shows the MSE score for $K$, the number of latent classes, varying from 10 to 50 and for six datasets. One can see that for the different methods, the result is relatively stable, in a lesser extent on the networks hep\_th and fb\_uc. Thus, the choice of this parameter has a limited impact on the performances of the models. Nevertheless, it should be notice that for $K$ equals  to 50, EPM has not been able to handle the datasets.

\textbf{CL: Pbl axe 2 indique Wsim et non MSE}

%%%%%%%%%%%%%%%%%%%%%%%%%%%%%%%%%%%%%%%%%%%%%%%%%%%%%%%%
%%%% Timing performance
%%%%%%%%%%%%%%%%%%%%%%%%%%%%%%%%%%%%%%%%%%%%%%%%%%%%%%%%
\begin{table*}[t]
\centering
	\caption{Comparison of models in terms of the inference time, in hour, on different training sets for $K=10$.}

\resizebox{\textwidth}{!}{
\begin{tabular}{lllllllll}
\hline
              & MMSB-scvb          & WMMSB-bg           & EPM                          & SBM-ai               & WSBM-ai-n        & WSBM-ai-p          & SBM-gt                  & WSBM-gt                  \\
\hline                                                                                                                                                                  
astro-ph      & 0.09 $\pm$ 0.02    & 0.07 $\pm$ 0.02    & \textbf{1.08}  $\pm$ 0.0     & 0.08 $\pm$ 0.01      & 0.07 $\pm$ 0.01  & 0.08 $\pm$ 0.02    & 0.01 $\pm$ 0.0          & 0.03 $\pm$ 0.01          \\
enron         & 1.52 $\pm$ 0.62    & 1.29 $\pm$ 0.61    & \textbf{25.03} $\pm$ 0.03    & 1.13 $\pm$ 0.08      & 1.04 $\pm$ 0.05  & 1.00 $\pm$ 0.16    & 0.10 $\pm$ 0.01         & 0.22 $\pm$ 0.01          \\
fb\_uc         & 0.01 $\pm$ 0.0    & 0.02 $\pm$ 0.01    & \textbf{0.02}  $\pm$ 0.0     & 0.01 $\pm$ 0.0       & 0.01 $\pm$ 0.01  & 0.01 $\pm$ 0.0     & 0.00 $\pm$ 0.0          & 0.01 $\pm$ 0.0           \\
hep-th        & 0.05 $\pm$ 0.01    & 0.08 $\pm$ 0.04    & \textbf{0.26}  $\pm$ 0.01    & 0.02 $\pm$ 0.0       & 0.02 $\pm$ 0.01  & 0.02 $\pm$ 0.0     & 0.00 $\pm$ 0.0          & 0.01 $\pm$ 0.0           \\
wiki-link     & 2.15 $\pm$ 0.5     & 1.67 $\pm$ 0.34    & \textbf{25.10} $\pm$ 0.03    & 1.38 $\pm$ 0.16      & 1.48 $\pm$ 0.08  & 1.58 $\pm$ 0.33    & 0.29 $\pm$ 0.01         & 0.63 $\pm$ 0.05          \\
moreno\_names  & 0.02 $\pm$ 0.02   & 0.01 $\pm$ 0.01    & \textbf{0.02}  $\pm$ 0.0     & 0.01 $\pm$ 0.0       & 0.01 $\pm$ 0.01  & 0.01 $\pm$ 0.0     & 0.01 $\pm$ 0.0          & 0.01 $\pm$ 0.0           \\
digg-reply    & 0.51 $\pm$ 0.3     & 0.71 $\pm$ 0.27    & \textbf{4.69}  $\pm$ 0.04    & 0.15 $\pm$ 0.01      & 0.14 $\pm$ 0.01  & 0.15 $\pm$ 0.01    & 0.02 $\pm$ 0.01         & 0.06 $\pm$ 0.01          \\
prosper-loans & 0.87 $\pm$ 0.24    & 1.71 $\pm$ 0.78    & \textbf{25.13} $\pm$ 0.03    & 1.64 $\pm$ 0.09      & 1.62 $\pm$ 0.1   & 1.64 $\pm$ 0.2     & 1.38 $\pm$ 0.09         & 3.04 $\pm$ 0.22          \\
slashdot      & 1.27 $\pm$ 0.24    & 1.51 $\pm$ 0.48    & \textbf{12.98} $\pm$ 1.26    & 0.39 $\pm$ 0.03      & 0.36 $\pm$ 0.03  & 0.39 $\pm$ 0.04    & 0.04 $\pm$ 0.01         & 0.08 $\pm$ 0.01          \\
\hline
\end{tabular}
}


\label{table:time}
\end{table*}
Table \ref{table:time} presents the inference time of the models on the different dataset in hours. The models MMSB-scvb and WMMSB-gt are implemented in Python, SBM-ai-b/WSBM-ai-n/WSBM-ai-p in Matlab and SBM-gt and WSBM-gt in C++ (with a Python wrapper \cite{peixoto_graph-tool_2014}), which must be considered when comparing the timing. Furthermore, a hard limit  was set at 25 hours for the inference time to limit the computing cost. We can observe that SBM-gt is the fastest model while EPM is the slowest. This last one can take more than 20 times the inference duration of MMSB and WMMSB; which confirms its incapacity to handle the networks with $K$ equals 50 as seen previously. Notably, the inference of MMSB and WMMSB with the SCVB inference scheme fits millions of edges in less than 2 hours, while it is implemented in Python, which is very satisfying and allows to expect better performances  with optimized implementations for this inference scheme.



%%%%%%%%%%%%%%%%%%%%%%%%%%%%%%%%%%%%%%%%%%%%%%%%%%%%%%%%
%%%% Convergence of *MMSB SCVB inference
%%%%%%%%%%%%%%%%%%%%%%%%%%%%%%%%%%%%%%%%%%%%%%%%%%%%%%%%
Figure \ref{fig:conv_entropy} shows the evolution of the log-likelihood for the MMSB-based models on the test set for enron, slashdot and proper-loans datasets. We used three different sets for the hyperparameters shape $r$ and scale $p$ of WMMSB. Regardless of the values of these hyperparameters, one can observe that the augmented model WMMSB-bg is less prone to overfitting, usually converges to a better solution and only needs a small proportion of the total number $N^2$ of edges to do so.
\textbf{CL : detail sur les 3 ensembles a preciser ?} 

\begin{figure*}[h]
\centering
	%

\begin{subfigure}
     \centering
         \includegraphics[width=0.23\textwidth]{fig/astro-ph_fig__entropy}
\end{subfigure}
\begin{subfigure}
         \centering
      \includegraphics[width=0.23\textwidth]{fig/digg-reply_fig__entropy}               
\end{subfigure}                                                                          
\begin{subfigure}                                                                        
         \centering                                                                      
      \includegraphics[width=0.23\textwidth]{fig/hep-th_fig__entropy}
\end{subfigure}                                                                          
\begin{subfigure}                                                                        
     \centering                                                                          
         \includegraphics[width=0.23\textwidth]{fig/enron_fig__entropy}
\end{subfigure}
\begin{subfigure}
         \centering
      \includegraphics[width=0.23\textwidth]{fig/prosper-loans_fig__entropy}
\end{subfigure}                                                             
\begin{subfigure}                                                           
         \centering                                                         
      \includegraphics[width=0.23\textwidth]{fig/slashdot_fig__entropy}
\end{subfigure}                                                             
\caption{Log-likehood convergence for WMMSB and WMMSB-bg models on a test set containing 20\% of the edges of the networks. Three different sets of hyperparmeters are used for WMMSB.}


	\begin{subfigure}                                                                        
     \centering                                                                          
         \includegraphics[width=0.32\textwidth]{fig/enron_fig__entropy__}
\end{subfigure}
\begin{subfigure}
         \centering
      \includegraphics[width=0.32\textwidth]{fig/prosper-loans_fig__entropy__}
\end{subfigure}                                                             
\begin{subfigure}                                                           
         \centering                                                         
      \includegraphics[width=0.32\textwidth]{fig/slashdot_fig__entropy__}
\end{subfigure}                                                             
\caption{Log-likehood convergence for WMMSB and WMMSB-bg models on a test set containing 20\% of the edges of the networks. Three different sets of hyperparmeters are used for WMMSB.}


    \label{fig:conv_entropy}
\end{figure*}







