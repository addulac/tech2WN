\section{Introduction}


In social networks, the presence of a tie between two entities generally indicates that there is a relationship between them and in recent years, researchers proposed various methods to predict the presence or absence of such interaction. However, in many applications, it is the intensity of the relationship that is important. So, for example in epidemiology, it is not enough to know that two people have been in contact but it is also necessary to know the frequency of these contacts to evaluate whether there is a risk of contagion or not. Similarly, in the field of transport, it is not the fact that there is a motorway or an airline between two cities which allows to analyze the population flows between them, we have also  to estimate the number of vehicles or passengers between them. In the field of economy and finance, to know if a company risks being absorbed by another it is not enough to learn that the second took shares in the capital of the second but it is also necessary to know how much.  In these examples, the intensity of the relationship is generally modeled as a weight and the network is represented by a weighted graph.   Unfortunately, in practice it is often easier to observe the interaction than to measure it. 
%Such weights can also result from the observation of a temporal network at different dates and, in that case, the weight corresponds to the number of points in time where the link has been active.
For this reason, there has been plenty of works devoted to link prediction \cite{Liben2007,Zaki2011,Martinez2016} and less to weight prediction in social networks.\\
Like for link prediction, to solve the weight prediction problem, there are two main approaches, similarity-based and likelihood-based methods \cite{Lu2011}. The methods belonging to the first family assume that the similarity between two nodes is correlated with their interaction strength. For instance, using this hypothesis, Zhao \textit{et al.} propose to predict missing links and their weights by extending unweighted similarity indices to weighted ones \cite{Zhao2015}. Zhu \textit{et al.} also introduce a method where the estimation of the weights is based on neighbor sets \cite{Zhu2016}. Kumar \textit{et al.}, for their part, study the problem on signed social networks where links are labeled as positive or negative to indicate relationships such as approval/disapproval, like/dislike or trust/distrust and in that case, the weight takes  a negative or  positive value to evaluate the degree of opinion \cite{Kumar2016}. However, most often in valued networks, as in the examples given previously,  the  weights correspond to positive integers and it is the case studied in this paper.\\
The second approach formulates the problem in a probabilistic framework and includes extended versions of generative models, initially introduced to learn the network formation process. 
Among such generative models, stochastic block models (SBM) and their extensions through mixed-membership block models (MMBM)  have received particular attention \cite{ Karrer2011,airoldi2009mixed,iMMSB,fan2015dynamic} as they can account for the underlying classes that structure real-world networks and in particular social networks. Nevertheless, most models proposed so far are devoted to unweighted networks and, to our knowledge, only two models, in the stochastic block model family, have been proposed for weighted graphs: the latent block structure model of \cite{aicher2014learning} and the weighted stochastic block model of \cite{peixoto2018nonparametric}. These two models however suffer from the same drawback as standard stochastic block models, namely the fact that a node can belong to only one class, which is not realistic for many networks. Mixed-membership block models were specifically designed to overcome this limitation and we propose here a new mixed-membership block model adapted to weighted networks. One important aspect in designing a generative model for networks is to develop a scalable inference method so that the model can be applied on large networks. We rely in this study on collapsed variational inference coupled with stochastic variational inference to do so.\\
The remainder of the paper is organized as follows: Section~\ref{sec:model} presents the weighted mixed-membership models and Section~\ref{sec:inference} their inference; Section~\ref{sec:rl} describes related work; Section~\ref{sec:exps} illustrates the behavior of the proposed models on several real-world networks. Finally, Section~\ref{sec:concl} concludes the study.


%From social networks to protein interactions, from physics to linguistics, networks are one of the key representations for objects interacting with one another. The interest for modeling such networks has naturally increased with the availability of large datasets, and people have tried to design generative models to describe the formation of links between nodes. Among such generative models, stochastic block models and their extensions through mixed-membership block models have received particular attention \cite{airoldi2009mixed,iMMSB,fan2015dynamic} as they can account for the underlying classes that structure real-world networks and in particular social networks. Nevertheless, most models proposed so far are devoted to unweighted networks. To our knowledge, only two models, in the stochastic block model family, have been proposed for weighted graphs: the latent block structure model of \cite{aicher2014learning} and the weighted stochastic block model of \cite{peixoto2018nonparametric}. These two models however suffer from the same drawback as standard stochastic block models, namely the fact that a node can belong to only one class, which is not realistic for many networks. Mixed-membership block models were specifically designed to overcome this limitation and we propose here a new mixed-membership block model adapted to weighted networks. One important aspect in designing a generative model for networks is to develop a scalable inference method so that the model can be applied on large networks. We rely in this study on collapsed variational inference coupled with stochastic variational inference to do so.
%
%The remainder of the paper is organized as follows: Section~\ref{sec:model} presents the weighted mixed-membership models and Section~\ref{sec:inference} their inference; Section~\ref{sec:rl} describes related work; Section~\ref{sec:exps} illustrates the behavior of the proposed models on several real-world networks. Finally, Section~\ref{sec:concl} concludes the study.

%Especially in the machine learning literature, that focused on link prediction, dimensionality reduction and data exploration tasks. One of the main challenge in this area is to be able to handle massive networks that emerge from the web. In this paper, we focus on networks that underpin some kind of social relationship such as collaboration or communication networks. In this context, we propose an online learning algorithm that we derived for both binary and count edge covariate, within the framework of Mixed-Membership Stochastic Blockmodel (MMSB).

%%%% The type of networks that exists
%%Complex networks are graphs that are used to represents real world relationnal information. In computer science, a major network is the web that connects a large amount of data. There is a large diversity in the type of data that can be interconnected, which ca be a set of people in a social plateform, a set of documents linked with hyperlinks, communication networks of email  or more recently a graph of transaction encoded in a blockchain. Outside the web an other important networks is the one made of the scientific collaborations.
%
%%%% The Scalability problem => Sparse Network E/N**2 << 1
%%The complexity (time and memory) of batch algorithm are polynomial for graph. Thus, the need of online algorithm, able to update a model as data become available is fundamental for scaling strategy. This can because of the temporality of the data or more simply because the data don't fit in memory. Another source of diversity in networks is the support for labelled and dynamic networks. In this paper we study and propose an algorithm based on latent models with rich priors who scale for complex and massive networks, with labeled edges (weighted networks), and that can be adapted to model the exchangeability of sequences of binary networks (temporal networks).
