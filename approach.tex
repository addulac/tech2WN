
\section{Weighted Networks and Time exchangeability}

% dense - unrealistic
The link prediction models, in the literature \cite{goldenberg2010survey,lu2011link}, usually represent a network by a graph $G=(V,E)$, where $V$ is a set of vertex and $E$ a set of edges between the vertex. It can be represented by its adjacency matrix $(Y_{ij})_{(ij) \in E} \in \mathcal{Y}$. For a binary network, the support of the edges is $\mathcal{Y} = (0,1)$ to account respectively for the absence or the presence of an edge between two nodes. Formally the likelihood of such model is characterized by a Bernoulli density such that $y_{ij} \sim \mathrm{Bern}(\theta)$. If $\theta$ is a constant, this is an Erd\"o-R\'eniy model. Relaxing this unrealistic assumptions with a latent model gives the Stochastic Blockmodel, and more recently, by relaxing the hard assignment to blocks, gives the Mixed-Membership Stochastic Blockmodel (MMSB) \cite{MMSB}. Moreover all of those models are in a setting of binary networks and  static graphs which is formally traduced by an assumption of exchangeability. It means that the joint probability of a graph do not depend on the order of which we observes the nodes. %? $echangeability$

\textcolor{red}{\paragraph{cette sous section sur la sparsite a revoir et aligner avec les update SCVB (ce qsue je raconte dans sections rouge, est probablement incorrect (je laisse car reflexion en cours...)! The weighted model is still exchangeable ?}~\\
% sparse - real
The main limitation of such models is that they can't handle sparse networks, which is a corollary of the Aldous-Hoover theorem \cite{orbanz2015bayesian}. The models has been said to be misspecified. A way to alleviate this limitation is to weight edges instead of considering binary one. This weighting can be understood, in some way, as smoothing the networks. Additionally, a weighted network is a binary networks where information has bee added under the for of node labeling. The reason why sparsity could be handle this way is because, by considering a weighted networks, it makes all non-edges in a network (0 entries in the adjacency matrix) having a weak contribution to the degree distribution of associated node (see section \ref{todo}). Thus, the inference process can take advantage of this fact because, in sparse networks, most of the interactions are unrealized.}

\section{Model}


A powerful model for binary exchangeable networks is the Mixed Membership Stochastic Blockmodel (MMSB). We propose an extension of the MMSB

The proposed model is a weighted extension of MMSB, named WMMSB. The main design difference is that the likelihood is drawn from a Poisson distribution, and the correlations between the shared classes are drawn from independent Gamma distributions. The model first draw latent class membership for each node from a shared Dirichlet distribution. Then for each interactions,  nodes draw a single membership. The observation level is then draw from Poisson distribution taking its rate parameter according to two classes that summarize the interaction between the two underlying nodes. The observation corresponds to the strength of the relationship between two classes. The generative model (along  with the graphical model) is summarized as follows:

\begin{figure}[h]
    \hspace*{-3.7em}
\begin{minipage}[h]{0.3\linewidth}
\begin{align*}
	&\textrm{For each } i \in \{1, .., N\}  \\
	&\qquad\bm{f}_i \sim \textrm{Dir}(\alpha)\\
	&\textrm{For each }  (m,n) \in \{1,..,K\}^2 \\
	&\qquad\phi_{mn} \sim \mathrm{Gamma}(\eta,\theta)\\
	&\textrm{For each } (i,j) \in V^2 \\
	&\qquad z_{i \rightarrow j} \sim \mbox{Cat}(\bm{f}_i)\\
	&\qquad z_{i \leftarrow j} \sim \mbox{Cat}(\bm{f}_j)\\
    &\qquad y_{ij} \sim \mathrm{Poisson}(\phi_{z_{i \rightarrow j}z_{i \leftarrow j}})
\end{align*}
\end{minipage}
\begin{minipage}[h]{0.3\linewidth}
	\scalebox{0.88}{\begin{tikzpicture}
    %\begin{scope}[yshift=0.5cm]
  % Define nodes
  \node[obs]                      (y) {$y_{ij}$};
  \node[latent, left=1.2cm of y] (zi) {$z_{ij}$};
  \node[latent, right=1.2cm of y] (zj) {$z_{ji}$};
  \node[latent, above= of y]    (ibp) {$\mat{F}$};;
  \node[latent, below= of y, yshift=-0.3cm]   (W) {$\mat{\Phi}$};
  \node[const, right=0.7cm of ibp]   (b) {$\alpha$};
  \node[const, right=0.7cm of W]   (sw) {($k,\theta$)};

  % Connect the nodes
  \edge {zi,zj,W} {y} ;
  \edge {ibp} {zi,zj} ;
  \edge {sw} {W} ; 
  \edge {b} {ibp} ; 

  % Plates
  \plate {yx} {(zj)(zi)(y)} {$N\times N$} ;
  %\end{scope}
\end{tikzpicture}
}
\end{minipage}
    \caption{Generative models and Bayesian graph of WMMSB.}
\end{figure}

Note that the group membership of each node is context dependent. That is, each node may assume different membership when interacting or being interacted with by different peers. Statistically, each node is an admixture of group-specific interactions. The two sets of latent group indicators are denoted by $Z_\rightarrow = \{z_{p\rightarrow q} : p, q \in V\}$ and $Z_\leftarrow = \{z_{p\leftarrow q} : p, q \in V\}$. Also note that the correlation between classes need not be equal. Equality may be enforced when modeling symmetric interactions. \cite{goldenberg2010survey}.

%In the rest of this paper  we will note the set of hyperparameters as $\H = (\alpha, \theta, k)$.

For simplicity, the marginal likelihood (or evidence), takes the following form:
\begin{equation*}
    \p(Y | \alpha, \Phi) = \int_F \sum_Z \p(Y| Z, \Phi) \p(Z|F) \p(F| \alpha) dF
\end{equation*}

Furthermore, observed links are conditionally independent given the parameters for nodes ($F$) and for classes ($\Phi$) and by marginalizing over the class membership. Thus the likelihood can be written as a matrix decomposition :
\begin{equation*}
P(Y | F, \Phi) = \prod_{i,j \in V^2} f_i \Phi f_j^T
\end{equation*} 

Unfortunately the evidence is intractable, and conducts the practitioner to resort to approximate inference. In the next section, we propose a Stochastic Collapsed Variational inference method.





\section{Inference}


\subsection{Collapsed Variational Inference}

The Variational Bayes (VB) method is an inference method based on the approximation of the Bayesian model's parameters, $(F, \Phi, Z)$, with free variational parameters $(\nu, \epsilon, \gamma)$ usually taken in the same family than the true parameters. It consists of an iterating algorithm which updates the variational parameters by maximizing the lower bound arising from the Jensen Inequality. This is equivalent to  minimize the Kullback-Leibler divergence between the variational (posterior) distribution over the variational parameters denoted $q(F, \Phi, Z | \nu, \epsilon, \gamma)$ (we will suppress reference to the variational parameters in the variational posterior for brevity) and the true (posterior) distribution $p(F, \Phi, Z | \tau)$ where $\tau = (\alpha, \eta, \theta)$ denotes the set of hyperparameters of the model. The evidence lower bound (ELBO) takes the following form :

\begin{align*}
    \log p(Y | \tau) &\geq \mathcal{L}(q(F, \Phi, Z)) \\
    &= E_q[\log p(Y, F, \Phi, Z | \tau)] - E_q[\log q(F, \Phi, Z)] \nonumber \nonumber
\end{align*}

In classical VB, the variational parameters are taken in the same family than the true parameters and by assuming that they are fully independent, this is know as the mean-field approximation :
\begin{align*}
q(F, \Phi, Z) &= \prod_{i\in V}q(f_i | \nu_i) \prod_{k,k' \in K^2}q(\phi_{kk'} | \epsilon_{kk'}) \\
    &\quad \prod_{i,j \in V^2}q(\zij, \zji | \gamma_{ij})
\end{align*}

In the Collapsed Variational Bayes (CVB), the assumption on the variational distribution is much weaker and therefore constitute a better approximation than standard VB \cite{teh2007collapsed}. We only assume that the $Z$ variables are mutually independent, then the variational posterior becomes : 
\begin{align*}
q(F, \Phi, Z) &= q(F,\Phi | Z) \prod_{(i,j)\in V^2}q(\zij, \zji | \gamma_{ij}) \\
	&= q(F, \Phi | Z)q(Z)
\end{align*}

The ELBO can be split by separating both independent contributions of the variational posterior. Since we do not restrict the form of $q(F, \Phi | Z)$, the ELBO maximum is achieved when the true posterior is reached at $p(F, \Phi | Y, Z,  \tau) = q(F, \Phi | Z)$. The ELBO simplifies in the following collapsed form :
\begin{align*}
    \mathcal{L}(q(Z)) &= \max_{q(F,\Phi|Z)} \mathcal{L}(q(F, \Phi|Z)q(Z)) \\
    &= E_q[\log p(Y,Z|\tau)] - E_q[q(Z)]
\end{align*}

The variational parameters are chosen in the same family of distribution than the true distribution, thus $q(\zij=k, \zji=k' | \gamma_{ij})$ are multinomial with parameters $\gamma_{ij}$.

Finally, the CVB update is obtained by maximizing the collapsed ELBO with respect to $\gamma_{ijkk'}$ and it takes the following general form :
\begin{equation} \label{eq:cvb_update}
    \gamma_{ijkk'} \propto \exp \left( E_{q(Z^{\neg ij})} [\log p(\zij=k, \zji=k' |Y, Z^{\neg ij}, \tau)] \right)
\end{equation}

Where the superscript $\neg ij$ means that the corresponding variables (or counts) are excluded.
This equation is not directly tractable, therefore it is possible to approximate equation \eqref{eq:cvb_update} by noting its relation with the collapse Gibbs update. Then,  the closed form expression for the CVB update can be obtained by using a central limit theorem approximation followed by a Taylor expansion.

The Collapse Gibbs Sampler (CGS) for both models are given by the following updates:

%\begin{align*} \label{eq:post_cgs}
%    p(\zij=k, \zji=k' | Y, Z^{\neg ij}, \tau) \propto  \quad & p(y_{ij} | Y^{\neg ij}, Z^{\neg ij}, \zij=k, \zji=k', \tau) \nonumber\\
%    & p(\zij=k, \zji=k' | Z^{\neg ij}, \tau)
%\end{align*}
\begin{align} \label{eq:post_cgs}
    &p(\zij=k, \zji=k' | Y, Z^{\neg ij}, \tau) \propto  \nonumber\\
    & \quad p(\zij=k, \zji=k' | Z^{\neg ij}, \tau) \nonumber \\
    & \quad p(y_{ij} | Y^{\neg ij}, Z^{\neg ij}, \zij=k, \zji=k', \tau)
\end{align}

Both MMSB and WMMSB models had in common the prior over the latent classes which characterise the mixed-membership effect. The update for the left hand part of equation \eqref{eq:post_cgs} represents a reinforcement effect on latent classes membership :

\begin{align*}
    & p(\zij=k, \zji=k' | Z^{\neg ij}, \tau) \propto  \nonumber\\
    & (n^{F\neg j}_{ik} + \alpha_k) (n^{F\neg i}_{jk} + \alpha_{k'}) 
\end{align*}


Both models difference arise from their kernels that are respectively Bernoulli-Beta and Poisson-Gamma distributed for MMSB and WMMSB. The righ part of equation \eqref{eq:post_cgs} are given by the updates in table \ref{table:cgs} :

%\paragraph{MMSB :}
%%    \item MMSB : {\setlength{\mathindent}{0cm} \begin{align*} \quad & p(y_{ij} | Y^{\neg ij}, Z^{\neg ij}, \zij=k, \zji=k', \tau) = \nonumber \\ & \quad  \frac{ n^{\Phi\neg ij}_{y_{ij}kk'} + \lambda_{y_{ij}}}{n^{\Phi\neg ij}_{\bm{.}kk'} + \lambda_{\bm{.}}}\end{align*}}
%    \begin{align*} \quad & p(y_{ij} | Y^{\neg ij}, Z^{\neg ij}, \zij=k, \zji=k', \tau) = \nonumber \\ & \quad  \frac{ n^{\Phi\neg ij}_{y_{ij}kk'} + \lambda_{y_{ij}}}{n^{\Phi\neg ij}_{\bm{.}kk'} + \lambda_{\bm{.}}}\end{align*}
%
%\paragraph{WMSB :}
%\begin{align*} &\quad p(y_{ij} | Y^{\neg ij}, Z^{\neg ij}, \zij=k, \zji=k', \tau) = \nonumber\\& \quad  NB(y_{ij}; n^{Y\neg ij}_{kk'} + \eta, \frac{1}{n^{\Phi\neg ij}_{\bm{.}kk'} + \theta + 1} )\end{align*}
%
\begin{table}[h] \label{table:cgs}
\caption{Gibbs updates for the MMSB and WMMSB kernels.}

    \begin{tabular}{ll}
    \hline
    CGS update    & $p(y_{ij} | Y^{\neg ij}, Z^{\neg ij}, \zij=k, \zji=k', \tau)$ \\
    \hline
    MMSB  & $\frac{ n^{\Phi\neg ij}_{y_{ij}kk'} + \lambda_{y_{ij}}}{n^{\Phi\neg ij}_{\bm{.}kk'} + \lambda_{\bm{.}}}$ \\
    WMMSB & $NB(y_{ij}; n^{Y\neg ij}_{kk'} + \eta, \frac{1}{n^{\Phi\neg ij}_{\bm{.}kk'} + \theta + 1} )$ \\
    \hline
    \end{tabular}
\end{table}


Where $NB(x;a, b)$ represents the pdf of a Negative Binomial distribution parametrized by $(a,b)$.

The collapse Gibbs Samplers only need to update the various counts $n^*$ that emerge from the conjugacy of prior in the *MMSB models. The counts represents the following quantities :
%\begin{align*}
%    &n^{F}_{ik} = \sum_j \delta(\zij=k) + \delta(z_{j\leftarrow i}=k) \qquad\qquad\qquad n^{Y}_{kk'} = \sum_{ij} y_{ij}\delta(\zij=k, \zji=k') \\
%    &n^{\Phi}_{xkk'} = \sum_{ij} \delta(y_{ij}=x, \zij=k, \zji=k') \qquad  n^{\Phi}_{\bm{.}kk'} = \sum_x n^{\Phi}_{xkk'}
%\end{align*}
\begin{align*}
&n^{F}_{ik} = \sum_j \delta(\zij=k) + \delta(z_{j\leftarrow i}=k) \\  
&n^{Y}_{kk'} = \sum_{ij} y_{ij}\delta(\zij=k, \zji=k') \\
&n^{\Phi}_{xkk'} = \sum_{ij} \delta(y_{ij}=x, \zij=k, \zji=k') \\
&n^{\Phi}_{\bm{.}kk'} = \sum_x n^{\Phi}_{xkk'}
\end{align*}

In order to compute equation \eqref{eq:cvb_update}, one can approximate the variational distribution with a zero Taylor expansion under a Gaussian approximation. 

By the central limit theorem, one can compute the expectations of the counts statistics under the variational distribution, that we will refer as $N^* \approx E_q[n^*]$. They are given by a sum of independent variables, where each token is the result of a Bernoulli trial with mean $\gamma_{ijkk'}$. The variational counts statistics can be obtained as follow:

\begin{align} \label{eq:stat_cvb}
    &N^{F}_{ik} = \sum_{j, k'} \gamma_{ijkk'} \qquad\quad  N^{Y}_{kk'} = \sum_{ij} y_{ij}\gamma_{ijkk'} \\
    &N^{\Phi}_{xkk'} = \sum_{ij:y_{ij}=x} \gamma_{ijkk'} \qquad  N^{\Phi}_{\bm{.}kk'} = \sum_x N^{\Phi}_{xkk'}
\end{align}

Under this Gaussian approximation, we can apply a Taylor expansion of equation \eqref{eq:cvb_update}. For simplicity and effectiveness \cite{asuncion2009smoothing}, we propose a zero order approximation, which is know as the CVB0 algorithm and gives the following updates :

\paragraph{MMSB :}
\begin{align*}
\gamma_{ijkk'} \propto  \frac{ N^{\Phi\neg ij}_{y_{ij}kk'} + \lambda_{y_{ij}}}{N^{\Phi\neg ij}_{\bm{.}kk'} + \lambda_{\bm{.}}} (N^{F\neg j}_{ik} + \alpha_k) (N^{F\neg i}_{jk} + \alpha_{k'}) 
\end{align*}

\paragraph{WMMSB :}
\begin{align*}
\gamma_{ijkk'} &\propto NB(y_{ij}; n^{Y\neg ij}_{kk'} + \eta, \frac{1}{n^{\Phi\neg ij}_{\bm{.}kk'} + \theta + 1} ) \\
&\quad (N^{F\neg j}_{ik} + \alpha_k) (N^{F\neg i}_{jk} + \alpha_{k'}) 
\end{align*}

%%% Equation don't fit in table.
%\begin{table}[h] \label{table:cvb}
%\caption{CVB updates for the MMSB and WMMSB kernels.}
%
%    \begin{tabular}{ll}
%    \hline
%    CVB update    & $\gamma_{ijkk'}$ \\
%    \hline
%    MMSB  & $\frac{ N^{\Phi\neg ij}_{y_{ij}kk'} + \lambda_{y_{ij}}}{N^{\Phi\neg ij}_{\bm{.}kk'} + \lambda_{\bm{.}}} (N^{F\neg j}_{ik} + \alpha_k) (N^{F\neg i}_{jk} + \alpha_{k'})$ \\
%    WMMSB & $NB(y_{ij}; n^{Y\neg ij}_{kk'} + \eta, \frac{1}{n^{\Phi\neg ij}_{\bm{.}kk'} + \theta + 1} )(N^{F\neg j}_{ik} + \alpha_k) (N^{F\neg i}_{jk} + \alpha_{k'})$ \\
%    \hline
%    \end{tabular}
%\end{table}


One can remark the similarity with the Collapse Gibbs update. But as the inference is deterministic, it is possible to derive a stochastic gradient in order to prevent batch learning and to scale to large datasets.

\subsection{Stochastic Variational Inference}

We propose to derive update for a Stochastic Collapsed Variational Inference (SCVB) for the *MMSB models. It is closely related to The Stochastic Variational Inference (SVB)  where online inference is made possible with variational Inference in the exponential family and to \cite{hoffman2013stochastic} which propose general online E-M algorithm for latent model \cite{cappe2009line}. This approach is useful for the two following reasons :

\begin{itemize}
    \item Scale to large datasets : The time complexity becomes linear with the size of the datasets because the algorithm can work with minibatch of an arbitrary size. The memory complexity is drastically save, because only the minibatch content need be loaded to fit a model against the quadratic complexity usually met in networks.
    \item Dynamic scenario : The size of the minibatch can be arbitrary chosen, and could be use in streaming learning scenario, by relaxing the assumptions made on the step size if the Stochastic gradient.
\end{itemize}

The CVB algorithm is resumed by the statistics $N^*$ (see equations \eqref{eq:stat_cvb}). It means that the model parameter can be recovered from those statistics :

\begin{align*}
    \hat f_{ik} &= \frac{N^{F}_{ik} + \alpha_k }{2|V| + \alpha_{\bm{.}}} \\
    \hat \phi_{kk',x} &= \frac{ N^{\Phi}_{x, kk'} + \lambda_x}{N^{\Phi}_{\bm{.}, kk'} + \lambda_{\bm{.}}}  \qquad \mathrm{for}\quad\mathrm{MMSB}\\
    \hat \phi_{kk',x} &= NB(x; N^Y_{kk'} + \eta, \frac{1}{N^{\Phi}_{\bm{.}kk'} + \theta + 1} ) \qquad \mathrm{for}\quad\mathrm{WMMSB}
\end{align*}

For one observations, global statistics are updated by computing the natural gradient of the objective under variational parameter and given by the following steps :
\begin{itemize}
    \item Maximization : The current state is approximately maximized from the update of $\gamma_{ij}$ which optimize the ELBO for the current token,
    \item Expectation : The expected count statistics are updated which propagates the new observations to the noisy gradient. This propagation is controlled by a gradient step size $\rho^{*}_t$ such that $N^{*, t} = (1-\rho_t)N^{*,t-1} + \rho_t \overline N^{*,t}$
\end{itemize}

Where $\overline N$ is the expected statistics computed as if all observations where drawn as the current variational parameters $\gamma_{ij}$. That is the corresponding  expectation for an observations $(ij)$  are given for the different statistics :
\begin{itemize}
    \item $\overline N^{F}_i = R_i \gamma_{ij}$ where $R_i=2|V_i|$ is the total number of relations for a node $i$ in a graph,
    \item $\overline N^{\Phi} = R Y^{(ij)}$ where $R$ is the total number of relations in the network, for a directed graph is $R=|V|^2$, and $Y^{(ij)}$ is $|\mathcal{Y}| K^2$ tensor such that the matrix entries corresponding to the index equal to $y_{ij}$ are set to $\gamma_{ij}$ and the other ones to zeros.
    \item $\overline N^Y = Ry_{ij}\gamma_{ij}$, similar statistics than  $\overline N^{\Phi}$ for weighted networks.
        
\end{itemize}

Todo :
\begin{itemize}
    \item gradient step constraint for convergence
    \item (reflexion) what does  it means in term of convergence and step gradient constraint,  if I want to forget old data, make a global burstiness effect (a sparse network) ??
\end{itemize}

Thus our algorithm is very similar than the one use by \cite{foulds2013stochastic} for their SCVB algorithm adapted for LDA and text data.









\section{Experiments}

%WMMSB compare it with LDA (a mask)i vs a bipartite networks (Document)
%D (doc) x W (Voc) -- entry count
%
%the bag of word VS the networks representation of document
%* it adds a parameter the document modelization !!!!!!!
