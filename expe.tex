\section{Experimental validation}
\label{sec:exps}

%
% Corpus
%


We experimented our models and inference on several real world, either directed or undirected, networks. Theirs statistics and properties are summarized in table \ref{table:corpus}. The detailed descriptions are available in the online Koblenz network collection\footnote{http://konect.uni-koblenz.de/}.

\begin{table}[h]
\bgroup
\def\arraystretch{1} % 1 is the default, change whatever you need
	
\caption{Datasets networks used to train the models. Type A is for co-authorship, type C is for communication, type H is for hyperlinks and type L is for lexical network.}

\begin{tabular}{lrrrrcrrrr}
%\Xhline{2\arrayrulewidth}
\toprule
 Datasets     &   Nodes &   Edges &   Density & Directed  &    Diameter &   \multicolumn{3}{c}{Weights}  	& type     \\
 \cmidrule(l){7-9}  &   &   	  &   		  & 		  &  		   	&  mean & std  & max             \\
%\hline
\midrule
 astro-ph      &   16706 &  121251 &     0.001    & False    &       14 &   1.8& 3.3 & 306      & A  \\
 cond-mat      &   16726 &   47594 &     0.000    & False    &       18 &   3.1& 7.2& 544      & A  \\
 hep-th        &    8361 &   15751 &     0.000    & False    &        1 &   5.2& 16& 1226      & A  \\
 netscience    &    1589 &    2742 &     0.002    & False    &        2 &   2.2& 1.9 &33       & A  \\
 manufacturing &     167 &    5783 &     0.209    & True     &        3 &  14.3& 44.9&1458    & C  \\
 fb\_uc        &    1899 &   20296 &     0.006    & True     &        4 &   2.8& 4.7& 98       & C  \\
 enron         &    87273  & 320154   &  0.0000   & True     &       15 &   3.4& 12.4&3904    & C  \\
 wiki-link     &  100312   & 887426   & 0.0001    & True     &       14 &   1.7& 3.0& 185      & H  \\
%\Xhline{2\arrayrulewidth}
\bottomrule
\end{tabular}


\egroup
\label{table:corpus}
\end{table}



%
% Evaluation method (testing set and mesure)
%
\paragraph{Performance Analysis} To evaluate the performance of the models and to illustrate the advantage of the online inference, we designed an experiment based on different size of training sets to test the impact of the amount of data on the inference. The evaluation rely on a AUC-ROC metric on the missing link prediction task. For weighted models, we considered the probability that an edge exists between two unobserved nodes (i,j) as the probability that they have at least one edge count by $p(y_{ij} \geq 1 | \Thetah, \Phih) = 1 - \sum_{kk'} \thetah_{ik} \thetah_{jk'} e^{\phih_{kk'}}$.

For all the dataset, we  built a test set by extracting randomly 20 percent of the edges of the networks and about the same amount of non-links. The remaining data constitute the "full" training set. Then, we subsamples the full training set in order to obtain smaller sub training set (subgraphs) containing different proportions of the edges (i.e 1, 5, 10, 20, 30, 50, and 100 percent. The former corresponding to the full training set.). Note that we ensure that all the sub training set are inclusive. We repeated this settings 10 times with different seed, to cross validate our results, and report mean/std results.

% stopping citerion
For the stopping criteria of the inference, we further held-out 10 percent of the remaining (sub) training set used as a validation set. After each minibatch iteration, we computed the log-likelihood on this validation set. When the increase of the log-likelihood on average (on the last 20 measures) is less that 0.001, the inference is stopped. The log-likelihood of a given set of observations $\D_{test}$  is computed as follows:
\begin{equation*}
\log p(\D_{test}) = \sum_{i,j \in \D_{test}} \log p(y_{ij} | \phih_{kk'}) p(k|\thetah_i) p(k'|\thetah_j)
%\log p(\D_{test}) = \sum_{i,j \in \D_{test}} \log p(y_{ij} | \phih_{kk'}) p(k|\thetah_i) p(k'|\thetah_j)
\end{equation*}

For all our models, we set the gradient step parameters to $\tau=1024$ and $\kappa=0.5$, the burn-in period to $T_{burnin}=150$ and for the stratified sampling, we fix the size of the non-edges set to  $M=50$. For MMSB we set the hyperparameters $\lambda_0=\lambda_1=0.1$ and for WMMSB, the shape and scale parameters $r$ et $p$ were fixed to 1. The number of latent classes was fixed to $K=10$ for all models and the latent-class hyperparameters to $\alpha_k=\frac{1}{K}$.


We compared our models against two baselines, namely SBM and WSBM that are the non hierarchical counterpart of MMSB and WMMSB and constitutes standard approach for link prediction. For both baselines, we used the microcanonical stochastic block model implementation of \cite{peixoto2018nonparametric} that introduces an efficient MCMC inference method for the stochastic block models family. This implementation can be found in the graph-tool framework \cite{peixoto_graph-tool_2014}. The number of classes was again set to $K=10$ for both models.


\textcolor{white}{ % trick to make it invisible, only in the footpage.
\footnote{\url{http://konect.uni-koblenz.de/networks/ca-AstroPh}. We used the cleaned version available in the  graph-tool framework.}
\footnote{\url{hhttp://konect.uni-koblenz.de/networks/ca-cit-HepTh}. We used the cleaned version available in the  graph-tool framework.}
\footnote{\url{hhttp://konect.uni-koblenz.de/networks/moreno_names}}
\footnote{\url{hhttp://konect.uni-koblenz.de/networks/radoslaw_email}}
\footnote{\url{hhttp://konect.uni-koblenz.de/networks/opsahl-ucsocial}}
\footnote{\url{hhttp://konect.uni-koblenz.de/networks/munmun_digg_reply}}
\footnote{\url{hhttp://konect.uni-koblenz.de/networks/slashdot-threads}}
\footnote{\url{hhttp://konect.uni-koblenz.de/networks/enron}}
\footnote{\url{hhttp://konect.uni-koblenz.de/networks/link-dynamic-simplewiki}}
\footnote{\url{hhttp://konect.uni-koblenz.de/networks/prosper-loans}}
}
\begin{figure}[h]
\centering
	

\begin{subfigure}
     \centering
         \includegraphics[width=0.32\textwidth]{fig/astro-ph__entropy@_roc_evo}
\end{subfigure}
\begin{subfigure}
         \centering
      \includegraphics[width=0.32\textwidth]{fig/digg-reply__entropy@_roc_evo}               
\end{subfigure}                                                                          
\begin{subfigure}                                                                        
         \centering                                                                      
      \includegraphics[width=0.32\textwidth]{fig/fb_uc__entropy@_roc_evo}
\end{subfigure}                                                                          
\begin{subfigure}                                                                        
         \centering                                                                      
      \includegraphics[width=0.32\textwidth]{fig/hep-th__entropy@_roc_evo}
\end{subfigure}                                                                          
\begin{subfigure}                                                                        
     \centering                                                                          
         \includegraphics[width=0.32\textwidth]{fig/enron__entropy@_roc_evo}
\end{subfigure}
\begin{subfigure}
         \centering
      \includegraphics[width=0.32\textwidth]{fig/prosper-loans__entropy@_roc_evo}
\end{subfigure}                                                             
\begin{subfigure}                                                           
         \centering                                                         
      \includegraphics[width=0.32\textwidth]{fig/slashdot__entropy@_roc_evo}
\end{subfigure}                                                             
\begin{subfigure}                                                           
         \centering                                                         
      \includegraphics[width=0.32\textwidth]{fig/wiki-link__entropy@_roc_evo}
\end{subfigure}                                                             
\begin{subfigure}                                                           
         \centering                                                         
      \includegraphics[width=0.32\textwidth]{fig/moreno_names__entropy@_roc_evo}
\end{subfigure}                                                             
\caption{Comparison of models in terms of AUC-ROC scores according to the percentage of edges used to train the models (from 1 to 100\%).}


   \label{fig:roc}
\end{figure}

The figure \ref{fig:roc}, represents the AUC performances of models depending on the proportion of the training edges used for the inference.  We can observe that the performances increase with the number of data used to learn, not surprisingly. This comportment is particularly pronounced for SBM and WSBM because their inference algorithm is designed to learn on batch dataset, while, our models converge faster to their best performance, due to the strong averaging step of the stochastic gradient descent, before they eventually overfit the training set.
WWMSB-bg outperforms WWMSB which highlight the importance of the gamma prior parameters in this model. Further, WMMSB-bg is competitive with the baselines and, when the number of missing links is high (i.e the training edges are lower or equal than 20 percent. See also table \ref{table:roc}), it outperforms other models. 
MMSB is competitive on some dataset (astro-ph, hep-th, moreno\_names) while it failed on the others. Interestingly, one can see than on the dataset where it fails (digg-reply, prosper-loans, slashdot and wiki-link), it has a very large variance. Furthermore, those dataset has a low density compared to astro-ph, hep-th and moreno\_names. Thus, this poor performance can be explained by the fact that the convergence of the model is very sensitive to the sampling choices during the online inference. Indeed, sparse network are known to exhibit a power law distribution in the degree distribution, which implies that some nodes don't have the same importance at all in this kind of networks. Nevertheless, the WMMSB and WMMSB-bg models are less sensitive to those networks and are still competitive, which show that taking into account the edge covariates makes the weighted models less sensitive to the minibatches sampling in sparse networks. Finally, it worth to mention that on dataset prosper-loans, which is the only networks classified as "Interaction" network in the Konnect repository, the baselines failed to learn the topology of the networks and MMSB barely exceed a random classifier, only WMMSB-bg succeed with its best performance above 0.75.



% Table Roc comments
\begin{table}
\centering
	
\caption{Comparison of MMSB, WMMSB-bg, SBM and WSBM in terms of AUC-ROC when using 10\% and 100\% of the training data.}

%\resizebox{12cm}{!}{
\resizebox{\textwidth}{!}{

\begin{tabular}{lllll|llll}
\toprule
&   \multicolumn{4}{c}{10\%} &   \multicolumn{4}{c}{100\%}   \\
\cmidrule(l){2-5} \cmidrule(l){6-9}   & MMSB & WMMSB-bg & SBM & WSBM & MMSB & WMMSB-bg & SBM & WSBM   \\
%\hline
\midrule
astro-ph      & \textbf{708} $\pm$ 3  & 700 $\pm$ 30           & 594 $\pm$ 16 & 586 $\pm$ 9           & \textbf{716} $\pm$ 11 & 710 $\pm$ 18          & 701 $\pm$ 6           & 705 $\pm$ 5 \\
hep-th        & \textbf{617} $\pm$ 11 & 579 $\pm$ 12           & 480 $\pm$ 9  & 482 $\pm$ 26          & 675 $\pm$ 8           & 676 $\pm$ 8           & \textbf{779} $\pm$ 10 & 714 $\pm$ 7 \\
moreno\_names & 680 $\pm$ 72          & \textbf{707} $\pm$ 29  & 571 $\pm$ 29 & 594 $\pm$ 30          & 738 $\pm$ 33          & 739 $\pm$ 7           & \textbf{862} $\pm$ 7  & 859 $\pm$ 11 \\
fb\_uc        & 732 $\pm$ 127         & \textbf{827} $\pm$ 8   & 726 $\pm$ 20 & 788 $\pm$ 18          & 784 $\pm$ 140         & 850 $\pm$ 20          & \textbf{902} $\pm$ 2  & 896 $\pm$ 2 \\
digg\_reply   & 485 $\pm$ 178         & \textbf{651} $\pm$ 127 & 551 $\pm$ 47 & 582 $\pm$ 35          & 482 $\pm$ 204         & \textbf{744} $\pm$ 15 & 728 $\pm$ 26          & 717 $\pm$ 17 \\
slashdot      & 519 $\pm$ 193         & \textbf{820} $\pm$ 6   & 721 $\pm$ 66 & 732 $\pm$ 81          & 634 $\pm$ 181         & 791 $\pm$ 11          & 830 $\pm$ 16          & \textbf{834} $\pm$ 12  \\
enron         & 459 $\pm$ 289         & 875 $\pm$ 14           & 870 $\pm$ 80 & \textbf{923} $\pm$ 14 & 529 $\pm$ 256         & 835 $\pm$ 8           & 799 $\pm$ 20          & \textbf{853} $\pm$ 63  \\
wiki-link     & 491 $\pm$ 242         & 739 $\pm$ 73           & 848 $\pm$ 4  & \textbf{850} $\pm$ 4  & 432 $\pm$ 185         & 785 $\pm$ 8           & \textbf{925} $\pm$ 2  & 915 $\pm$ 3 \\
prosper-loans & 548 $\pm$ 284         & \textbf{752} $\pm$ 11  & 466 $\pm$ 57 & 455 $\pm$ 44          & 434 $\pm$ 274         & \textbf{727} $\pm$ 30 & 500 $\pm$ 4           & 504 $\pm$ 6 \\
\bottomrule
\end{tabular}
}

% 10 100
%\begin{tabular}{lllll|llll}
%\toprule
%&   \multicolumn{2}{c}{MMSB} &   \multicolumn{2}{c}{WMMSB-bg} &   \multicolumn{2}{c}{SBM} & \multicolumn{2}{c}{WSBM}   \\
%\cmidrule(l){2-3} \cmidrule(l){4-5} \cmidrule(l){6-7}\cmidrule(l){8-9}  & 10 & 100 & 10 & 100 & 10 &  100 & 10 & 100   \\
%%\hline
%\midrule                              
%astro-ph        &  \textbf{708} $\pm$ 3    & \underline{716} $\pm$ 11          & 700 $\pm$ 30            &  710 $\pm$ 18   &   594 $\pm$ 16   &  701 $\pm$ 6               &   588 $\pm$ 12           & 705 $\pm$ 5  \\
%hep-th          &  \textbf{617} $\pm$ 11   & 675 $\pm$ 8           & 579 $\pm$ 12            &  676 $\pm$ 8    &   480 $\pm$ 9    &  \underline{779} $\pm$ 1               &   497 $\pm$ 29            & 716 $\pm$ 9  \\
%moreno\_names   &  680 $\pm$ 72            & 738 $\pm$ 33          & \textbf{707} $\pm$ 29   &  739 $\pm$ 7    &   571 $\pm$ 29   &  \underline{862} $\pm$ 7               &   588 $\pm$ 25            & 862 $\pm$ 10 \\
%fb\_uc          &  732 $\pm$ 127           & 784 $\pm$ 14          & \textbf{827} $\pm$ 8    &  850 $\pm$ 20   &   726 $\pm$ 20   &  \underline{902} $\pm$ 2               &   787 $\pm$ 15            & 896 $\pm$ 2  \\
%digg\_reply     &  485 $\pm$ 178           & 482 $\pm$ 204         & \textbf{651} $\pm$ 127  &  744 $\pm$ 15   &   551 $\pm$ 47   &  728 $\pm$ 26              &   584 $\pm$ 34            & 714 $\pm$ 17 \\
%slashdot        &  519 $\pm$ 193           & 634 $\pm$ 181         & \textbf{820} $\pm$ 6    &  791 $\pm$ 11   &   721 $\pm$ 66   &  830 $\pm$ 16              &   699 $\pm$ 79            & \underline{833} $\pm$ 13 \\
%enron           &  459 $\pm$ 289           & 529 $\pm$ 256         & \textbf{875} $\pm$ 14   &  835 $\pm$ 8    &   870 $\pm$ 80   &  799 $\pm$ 20              &   866 $\pm$ 45            & \underline{842} $\pm$ 51 \\
%wiki-link       &  491 $\pm$ 242           & 432 $\pm$ 185         & 739 $\pm$ 73            &  785 $\pm$ 8    &   848 $\pm$ 4    &  \underline{925} $\pm$ 2               &   \textbf{853} $\pm$ 4    & 914 $\pm$ 4  \\
%prosper-loans   &  548 $\pm$ 284           & 434 $\pm$ 274         & \textbf{752} $\pm$ 11   &  \underline{727} $\pm$ 30   &   466 $\pm$ 57   &  500 $\pm$ 4               &   455 $\pm$ 44  	       & 505 $\pm$ 5  \\
%
%\bottomrule
%\end{tabular}
          
          
          





% 5 20 100

%&   \multicolumn{3}{c}{MMSB} &   \multicolumn{3}{c}{WMMSB-bg} &   \multicolumn{3}{c}{SBM} & \multicolumn{3}{c}{WSBM}     \\
%\cmidrule(l){2-4} \cmidrule(l){5-7} \cmidrule(l){8-10}\cmidrule(l){11-13}  & 5 & 20 & 100 & 5 & 20 & 100 & 5 & 20 & 100 & 5 & 20 & 100   \\

%astro-ph        &  686 $\pm$ 7    & 720 $\pm$ 8    & 716 $\pm$ 11   & 684 $\pm$ 25 & 690 $\pm$ 40  &  710 $\pm$ 18   &    505 $\pm$ 9  &  627 $\pm$ 5   &  701 $\pm$ 6   &  538 $\pm$ 31  &    626 $\pm$ 9   & 705 $\pm$ 5  \\
%hep-th          &  583 $\pm$ 13   & 655 $\pm$ 6    & 675 $\pm$ 8    & 558 $\pm$ 9  & 630 $\pm$ 21  &  676 $\pm$ 8    &   498 $\pm$ 6   &  506 $\pm$ 13  &  779 $\pm$ 1   &  513 $\pm$ 24   &    545 $\pm$ 29  & 716 $\pm$ 9  \\
%moreno\_names   &  674 $\pm$ 43   & 740 $\pm$ 18   & 738 $\pm$ 33   & 664 $\pm$ 39 & 697 $\pm$ 33  &  739 $\pm$ 7    &   478 $\pm$ 59  &  698 $\pm$ 25  &  862 $\pm$ 7   &  457 $\pm$ 38   &    709 $\pm$ 9   & 862 $\pm$ 10 \\
%fb\_uc          &  723 $\pm$ 109  & 728 $\pm$ 140  & 784 $\pm$ 14   & 790 $\pm$ 20 & 846 $\pm$ 11  &  850 $\pm$ 20   &   590 $\pm$ 43  &  846 $\pm$ 13  &  902 $\pm$ 2   &  679 $\pm$ 27   &    855 $\pm$ 7   & 896 $\pm$ 2  \\
%digg\_reply     &  491 $\pm$ 150  & 486 $\pm$ 195  & 482 $\pm$ 204  & 667 $\pm$ 73 & 723 $\pm$ 31  &  744 $\pm$ 15   &   528 $\pm$ 51  &  677 $\pm$ 13  &  728 $\pm$ 26  &  554 $\pm$ 22   &    659 $\pm$ 26  & 714 $\pm$ 17 \\
%slashdot        &  537 $\pm$ 177  & 538 $\pm$ 186  & 634 $\pm$ 181  & 801 $\pm$ 13 & 822 $\pm$ 6   &  791 $\pm$ 11   &   712 $\pm$ 66  &  775 $\pm$ 72  &  830 $\pm$ 16  &  731 $\pm$ 64   &    733 $\pm$ 28  & 833 $\pm$ 13 \\
%enron           &  555 $\pm$ 300  & 576 $\pm$ 301  & 529 $\pm$ 256  & 862 $\pm$ 15 & 876 $\pm$ 9   &  835 $\pm$ 8    &   900 $\pm$ 3   &  898 $\pm$ 39  &  799 $\pm$ 20  &  887 $\pm$ 48   &    916 $\pm$ 29  & 842 $\pm$ 51 \\
%wiki-link       &  389 $\pm$ 192  & 505 $\pm$ 230  & 432 $\pm$ 185  & 749 $\pm$ 43 & 725 $\pm$ 87  &  785 $\pm$ 8    &   517 $\pm$ 59  &  870 $\pm$ 4   &  925 $\pm$ 2   &  592 $\pm$ 159  &    871 $\pm$ 2   & 914 $\pm$ 4  \\
%prosper-loans   &  569 $\pm$ 300  & 622 $\pm$ 233  & 434 $\pm$ 274  & 736 $\pm$ 54 & 746 $\pm$ 28  &  727 $\pm$ 30   &   439 $\pm$ 97  &  503 $\pm$ 74  &  500 $\pm$ 4   &  472 $\pm$ 59   &    504 $\pm$ 39  & 505 $\pm$ 5  \\

                     
                     
                     
                     
                     
                     
                     
                     
                     
               
               
                                                                                                                                        
                     
                                                                                                                                           
                     
                     
                                                                                                                                           
                     
                     
                     
                                                                                                                     









\label{table:roc}
\end{table}




%
% Convergence commments
%
\paragraph {Convergence Analysis} We evaluated the convergence of WMMSB and WMMSB-bg models. Figure \ref{fig:conv_entropy} shows the convergence of the log-likelihood on a test set composed of 20 percent of the edges of the datasets and about the same amount of non-edges. We used three different set of the hyperparameters shape $r$ and scale $p$ for WMMSB. One can show that the augmented model is less prone to overfitting and that for most of the networks, it outperforms its counterpart terms of log-likelihood. Furthermore, one can see that the models requires a few proportions of the total $N**2$ potential edges to converge.

\begin{figure}[h]
\centering
	
\begin{subfigure}
     \centering
         \includegraphics[width=0.32\textwidth]{fig/astro-ph_fig__entropy}
\end{subfigure}
\begin{subfigure}
         \centering
      \includegraphics[width=0.32\textwidth]{fig/digg-reply_fig__entropy}               
\end{subfigure}                                                                          
\begin{subfigure}                                                                        
         \centering                                                                      
      \includegraphics[width=0.32\textwidth]{fig/fb_uc_fig__entropy}
\end{subfigure}                                                                          
\begin{subfigure}                                                                        
         \centering                                                                      
      \includegraphics[width=0.32\textwidth]{fig/hep-th_fig__entropy}
\end{subfigure}                                                                          
\begin{subfigure}                                                                        
     \centering                                                                          
         \includegraphics[width=0.32\textwidth]{fig/enron_fig__entropy}
\end{subfigure}
\begin{subfigure}
         \centering
      \includegraphics[width=0.32\textwidth]{fig/prosper-loans_fig__entropy}
\end{subfigure}                                                             
\begin{subfigure}                                                           
         \centering                                                         
      \includegraphics[width=0.32\textwidth]{fig/slashdot_fig__entropy}
\end{subfigure}                                                             
\begin{subfigure}                                                           
         \centering                                                         
      \includegraphics[width=0.32\textwidth]{fig/wiki-link_fig__entropy}
\end{subfigure}                                                             
\begin{subfigure}                                                           
         \centering                                                         
      \includegraphics[width=0.32\textwidth]{fig/moreno_names_fig__entropy}
\end{subfigure}                                                             
\caption{Log-likehood convergence for WMMSB and WMMSB-bg models. Three different set of hyper-parmeter are compared for WMMSB.}


    \label{fig:conv_entropy}
\end{figure}

While, our implementation is in pure python, the algorithm allows fast time convergence and is even comparable with SBM which is implemented in C within graph-tool. Furthermore, in practice, the algorithm exhibits a sublinear time complexity with the numbers of edges, as shown in figure \ref{table:time}.

\begin{table}[h]
\begin{tabular}{llllll}
\hline
                    & MMSB                  & WMMSB-bg              & SBM                  \\
\hline
astro-ph      & 361.165 $\pm$ 159.107    & 868.802 $\pm$ 870.965    & 44.646 $\pm$ 4.778      \\
hep-th        & 170.281 $\pm$ 17.246     & 226.887 $\pm$ 77.84      & 20.366 $\pm$ 4.127      \\
moreno\_names  & 44.821 $\pm$ 7.717       & 30.362 $\pm$ 11.811      & 5.358 $\pm$ 1.67        \\
fb\_uc         & 50.927 $\pm$ 17.694      & 66.921 $\pm$ 10.439      & 4.093 $\pm$ 0.907       \\
digg-reply    & 1803.779 $\pm$ 1100.227  & 2044.334 $\pm$ 1016.157  & 228.661 $\pm$ 81.611    \\
slashdot      & 4272.667 $\pm$ 1971.346  & 2867.998 $\pm$ 653.092   & 365.437 $\pm$ 82.475    \\
enron         & 4660.744 $\pm$ 2978.898  & 3544.711 $\pm$ 385.445   & 1313.841 $\pm$ 109.074  \\
wiki-link     & 13683.093 $\pm$ 6890.663 & 5592.084 $\pm$ 228.13    & 613.841 $\pm$ 113.047   \\
prosper-loans & 6595.175 $\pm$ 4612.049  & 12255.827 $\pm$ 4878.113 & 1182.409 $\pm$ 183.342  \\
\hline
\end{tabular}
\label{table:time}
\caption{Inference time in seconds.}
\end{table}



%
% WSim
%
%For the weighted models, we further measure the capacity to predict right edge counts with a $l_1$ distance between the real count of the test set and the expected count of the models 
%
%\begin{equation*}
%D_{l_1}(D_{test} ||  \{\Thetah, \Phih\}) = \sum_{i,j \in \D_{test}} | y_{ij} - \E[y_{ij}|\Thetah, \Phih] |
%\end{equation*}






